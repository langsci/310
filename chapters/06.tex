\documentclass[output=paper,colorlinks,citecolor=brown]{langscibook}
\author{Mark Hoff\affiliation{Ghent University} and Scott A. Schwenter\affiliation{The Ohio State University}}
\title{Settledness and morphosyntactic variation across Romance}
\abstract{Cross-linguistic research on tense/mood variation typically analyzes specific morphosyntactic environments separately and, as a result, a range of explanations have been put forth which apply, for example, only to conditionals or imperatives. We propose that an analysis based on the semantic-pragmatic notion of settledness (e.g. \citealt{ThomasonGupta1980}; \citealt{Kaufmann2002}, \citeyear{Kaufmann2005}) constitutes an essential first step toward accounting for tense/mood variation across different syntactic contexts in a way that unites disparate linguistic descriptions. Specifically, an analysis based on settledness captures the utility of using tense/mood to convey pragmatic information about speaker confidence in the realization of future eventualities. We analyze pairs of contrasting forms across several Romance languages: future-framed adverbials, imperatives, and conditionals. While only some of these forms make inherent reference to temporal distinctions such as present and future--some reflect modal and others person differences--we argue that the pragmatic motivation for the alternations is the same in all cases: they variably encode and reflect speaker judgments about the presumed settledness of the future eventuality under consideration, operationalized in terms of the related notions of speaker certainty, immediacy, and temporal specificity. The theoretical advantage of our account lies in the way it unites disparate descriptions of morphosyntactic phenomena across languages. Instead of analyzing them separately and by individual language, and treating them in terms of immediacy, epistemic certainty, or other diverse notions, analyzing this variation in terms of settledness brings these phenomena together under one overarching pragmatic concept with clear communicative motivations.}

\IfFileExists{../localcommands.tex}{%hack to check whether this is being compiled as part of a collection or standalone
   % add all extra packages you need to load to this file

\usepackage{tabularx,multicol,multirow}
\usepackage{url}
\urlstyle{same}

\usepackage{listings}
\lstset{basicstyle=\ttfamily,tabsize=2,breaklines=true}

\usepackage{langsci-basic}
\usepackage{langsci-optional}
\usepackage{langsci-lgr}
\usepackage{langsci-gb4e}
%    \let\eachwordone=\it % Ch 14, 18

\usepackage{jambox}
\usepackage{subfigure}
\usepackage{tablefootnote}
\usepackage[nameinlink, noabbrev]{cleveref}
\crefname{enumi}{example}{examples}

\usepackage{bbding}
%\usepackage{linguex}
\usepackage{stmaryrd}

\usepackage{tipa}
\let\ipa\textipa
\usepackage{vowel}
\newcommand{\BlankCell}{}
\usepackage{ot-tableau}

\usepackage{forest}
\useforestlibrary{linguistics}
\usepackage[noeepic]{qtree}
\usepackage{pstricks, pst-xkey, pst-jtree}
\usepackage{tikz-qtree}
\usepackage{tikz-qtree-compat}
\usepackage{tree-dvips}

\usepackage{lastpage}
\usepackage{hyperref}
\usepackage{xltxtra}

\usepackage{ragged2e}
%\usepackage{subcaption}
\usepackage{floatrow}
\usepackage{float}

\usepackage[normalem]{ulem} % Pour les textes barrés
\usepackage{ifthen} 

\usepackage{todonotes}

   \newcommand*{\orcid}{}

\makeatletter
\let\theauthor\@author
\makeatother

\papernote{\scriptsize\normalfont
    \theauthor.
    \titleTemp. 
    To appear in: 
    Chad Howe and Pilar Chamorro and Timothy Gupton and Margaret Renwick.
    Theory, Data, and Practice: Selected papers from the 49th Linguistic Symposium on Romance Language
    Berlin: Language Science Press. [preliminary page numbering]
}

% Workaround for subscripts with capital letters
\newcommand{\capsub}[1]{\ensuremath{_\text{#1}}}

% Chapter 10: Table-like presentation within example environment
% classical latin > {*}late latin > old french  earlier > later   gloss
\newcommand{\montanoboxi}[7]{\parbox{2cm}{#1} > {#2}\parbox{2cm}{#3} > \parbox{1.5cm}{\textit{#4}} \parbox{1.2cm}{#5}\ > \parbox{1.2cm}{#6} \parbox{1.5cm}{#7}}
% {*}latin > earlier OF [ipa] > early OF   gloss
\newcommand{\montanoboxii}[6]{{#1}\parbox{1.9cm}{\textit{#2}} > \parbox{1.3cm}{\textit{#3}} \parbox{2cm}{#4} \parbox{2cm}{#5} \parbox{1.9cm}{#6}}

% Chapter 5
\newcommand{\redc}[1]{\textcolor{red}{#1}}
\newcommand{\bluec}[1]{\textcolor{blue}{#1}}
\newcommand{\ajout}[1]{\textcolor{blue}{#1}}
\newcommand{\ajoutplus}[1]{\textcolor{cyan}{#1}}

\newcommand{\hachure}[9]{
% Parametres :
% Coordonnees bas gauche (2 parametres) : (#1,#2)
% Coordonnees haut droit (2 parametres) : (#3,#4)
% Orientation : #5
%   1 : diagonale de pente 1
%  -1 : diagonale de pente -1
%   0 : horizontal
%   2 : vertical
% Nombre de pas horizontaux : #6
% Epaisseur du trait : #7
% Couleur : #8 (ex. green)
% Atténuation couleur : #9 (ex. 30)
\pgfmathsetmacro{\N}{#6-1}
\pgfmathsetmacro{\A}{#1}
\pgfmathsetmacro{\B}{#2}
\pgfmathsetmacro{\C}{#3}
\pgfmathsetmacro{\D}{#4}
\pgfmathsetmacro{\I}{(#3-#1)/#6}
\pgfmathsetmacro{\J}{(#4-#2)/#6}
\ifthenelse{\equal{#5}{1}}{
  \foreach \n in {0,...,\N}
    \foreach \m in {0,...,\N}
      {
        \pgfmathsetmacro{\X}{\A + ((0 + \n) * \I)}
        \pgfmathsetmacro{\Y}{\B + ((0 + \m) * \J)}
        \pgfmathsetmacro{\U}{\A + ((1 + \n) * \I)}
        \pgfmathsetmacro{\V}{\B + ((1 + \m) * \J)}
        \draw[#8!#9,#7] (\X, \Y)--(\U, \V);
      } 
  }{}
\ifthenelse{\equal{#5}{-1}}{
  \foreach \n in {0,...,\N}
    \foreach \m in {0,...,\N}
      {
        \pgfmathsetmacro{\X}{\A + ((1 + \n) * \I)}
        \pgfmathsetmacro{\Y}{\B + ((0 + \m) * \J)}
        \pgfmathsetmacro{\U}{\A + ((0 + \n) * \I)}
        \pgfmathsetmacro{\V}{\B + ((1 + \m) * \J)}
        \draw[#8!#9,#7] (\X, \Y)--(\U, \V);
      } 
  }{}
\ifthenelse{\equal{#5}{0}}{
  \foreach \n in {0,...,\N}
    \foreach \m in {0,...,\N}
      {
        \pgfmathsetmacro{\X}{\A + ((0 + \n) * \I)}
        \pgfmathsetmacro{\Y}{\B + ((0 + \m) * \J)}
        \pgfmathsetmacro{\U}{\A + ((1 + \n) * \I)}
        \pgfmathsetmacro{\V}{\B + ((0 + \m) * \J)}
        \draw[#8!#9,#7] (\X, \Y)--(\U, \V);
      } 
  }{}
\ifthenelse{\equal{#5}{2}}{
  \foreach \n in {0,...,\N}
    \foreach \m in {0,...,\N}
      {
        \pgfmathsetmacro{\X}{\A + ((0 + \n) * \I)}
        \pgfmathsetmacro{\Y}{\B + ((0 + \m) * \J)}
        \pgfmathsetmacro{\U}{\A + ((0 + \n) * \I)}
        \pgfmathsetmacro{\V}{\B + ((1 + \m) * \J)}
        \draw[#8!#9,#7] (\X, \Y)--(\U, \V);
      } 
  }{}
}

%Définition d'un pattern de type hachure
% \usetikzlibrary{patterns}
% \makeatletter
% \tikzset{hatch distance/.store in=\hatchdistance,hatch distance=5pt,hatch thickness/.store in=\hatchthickness,hatch thickness=5pt}

% \pgfdeclarepatternformonly[\hatchdistance,\hatchthickness]{north east hatch}% name
%     {\pgfqpoint{-\hatchthickness}{-\hatchthickness}}% below left
%     {\pgfqpoint{\hatchdistance+\hatchthickness}{\hatchdistance+\hatchthickness}}% above right
%     {\pgfpoint{\hatchdistance}{\hatchdistance}}%
%     {
%         \pgfsetcolor{\tikz@pattern@color}
%         \pgfsetlinewidth{\hatchthickness}
%         \pgfpathmoveto{\pgfqpoint{-\hatchthickness}{-\hatchthickness}}       
%         \pgfpathlineto{\pgfqpoint{\hatchdistance+\hatchthickness}{\hatchdistance+\hatchthickness}}
%         \pgfusepath{stroke}
%     }
% \pgfdeclarepatternformonly[\hatchdistance,\hatchthickness]{north west hatch}% name
%     {\pgfqpoint{-\hatchthickness}{-\hatchthickness}}% below left
%     {\pgfqpoint{\hatchdistance+\hatchthickness}{\hatchdistance+\hatchthickness}}% above right
%     {\pgfpoint{\hatchdistance}{\hatchdistance}}%
%     {
%         \pgfsetcolor{\tikz@pattern@color}
%         \pgfsetlinewidth{\hatchthickness}
%         \pgfpathmoveto{\pgfqpoint{\hatchdistance+\hatchthickness}{-\hatchthickness}}
%         \pgfpathlineto{\pgfqpoint{-\hatchthickness}{\hatchdistance+\hatchthickness}}
%         \pgfusepath{stroke}
%     }
% \makeatother
%~~~~~~~~~~~~~~~~~~~~~~~~~~~~~~~~~~~~~


% Chapter 7
\newcommand\pef[1]{(\ref{#1})}

\newcommand{\subscript}[1]{\textsubscript}

   %% hyphenation points for line breaks
%% Normally, automatic hyphenation in LaTeX is very good
%% If a word is mis-hyphenated, add it to this file
%%
%% add information to TeX file before \begin{document} with:
%% %% hyphenation points for line breaks
%% Normally, automatic hyphenation in LaTeX is very good
%% If a word is mis-hyphenated, add it to this file
%%
%% add information to TeX file before \begin{document} with:
%% %% hyphenation points for line breaks
%% Normally, automatic hyphenation in LaTeX is very good
%% If a word is mis-hyphenated, add it to this file
%%
%% add information to TeX file before \begin{document} with:
%% \include{localhyphenation}
\hyphenation{
anaph-o-ra
Dor-drecht
%FFI2016-76045-P-AEI/-MINEICO/-FEDE
}

\hyphenation{
anaph-o-ra
Dor-drecht
%FFI2016-76045-P-AEI/-MINEICO/-FEDE
}

\hyphenation{
anaph-o-ra
Dor-drecht
%FFI2016-76045-P-AEI/-MINEICO/-FEDE
}

    \bibliography{localbibliography}
    \togglepaper[23]
}{}

\begin{document}
\maketitle

\section{Introduction}
Research on tense/mood contrasts in Romance, while abundant, has tended to focus on individual languages and, within these, often single syntactic constructions. To take an extreme case, it is easy to find studies which address the indicative/subjunctive contrast only in limited contexts of one variety of Spanish (e.g. noun clauses in Mexican Spanish), with little consideration of similar syntactic environments or of typologically related languages. In some sense, this limited scope has led to the propagation of disconnected and often divergent analyses of morphosyntactic variation. One such example is the disagreement among scholars regarding the primary factors conditioning variable future reference in Canadian and Hexagonal French (see e.g. \citealt{PoplackTurpin1999}, \citealt{Grimm2015}, and \citealt{GudmestadEdmondsDonaldsonCarmichael2018}). This issue is also seen in work on conceptually similar domains of grammar, such as conditionals (e.g. \citealt{Perini2002} for Brazilian Portuguese) and imperatives (e.g. \citealt{Johnson2013} for Argentine Spanish), where the opportunity to make meaningful connections across constructions is often missed (though see \citealt{Johnson2015} which relates voseo negative imperatives to uses of the same verb forms in other contexts).

In this paper, we describe tense/mood variation across multiple syntactic environments and in several Romance languages through the lens of a single semantic-pragmatic notion. Using both qualitative and quantitative data, we demonstrate that morphosyntactic variation in future-framed adverbial clauses (\textsc{ffa}s), future-framed conditionals (\textsc{ffc}s), and certain types of imperatives is sensitive (though to different degrees in each environment) to a contrast based on presumed settledness. Thus, these syntactic environments are united not only by their shared conceptual nature as referring to eventualities in the unrealized and unknowable future, but also by the fact that form choice in each is conditioned by a speaker’s confidence that a given eventuality is guaranteed to occur. Such an approach represents an important step toward cross-Romance generalization in that it appeals to a single, unified pragmatic motivation, rather than to a host of only moderately successful post-hoc explanations, to account for morphosyntactic variation.\footnote{See both \citet{PoplackTurpin1999} and \citet{PoplackDion2009} for invaluable discussion on the role grammars have played in the propagation of post-hoc readings of competing forms in an effort to account for variation.} Furthermore, and perhaps most importantly, this contrast is one that is simple, intuitive, and useful to speakers for accomplishing their communicative goals.

To begin, we present the syntactic environments in question, with the alternating ``standard'' and ``non-standard'' forms (see below) for each language. Each environment will be addressed more fully in its own section below.

\begin{exe}
\ex\label{ex:hoff:ffaexamples} \textbf{Future-framed adverbials}: \\ 
	\ex
	\begin{xlist}
\ex Italian\\Ti chiamo quando \textbf{arrivo} \textbf{[PI]} (cf. \textbf{arriverò} \textbf{[future indicative]}).
\ex Argentine Spanish\\Te llamo cuando \textbf{llego} \textbf{[PI]} (cf. \textbf{llegue} \textbf{[present subjunctive]}).
\ex Brazilian Portuguese\\Te ligo quando \textbf{chego} \textbf{[PI]} (cf. \textbf{chegar} \textbf{[future subjunctive]}).
\ex French\\Je t’appelle quand j’\textbf{arrive} \textbf{[PI]} (cf. \textbf{arriverai} \textbf{[future indicative]}).
\end{xlist}
`I'll call you when I arrive.'
\end{exe}

\begin{exe}
\ex\label{ex:hoff:ffcexamples} \textbf{Future-framed conditionals}: \\ 
	\ex
	\begin{xlist}
\ex Italian\\Se \textbf{finisco} \textbf{[PI]} (cf. \textbf{finirò} \textbf{[future indicative]}) presto di lavorare oggi, possiamo andare in spiaggia.
\ex Brazilian Portuguese\\Se eu \textbf{termino} \textbf{[PI]} (cf. \textbf{terminar} \textbf{[future subjunctive]}) de trabalhar mais cedo, a gente pode ir na praia.
\end{xlist}
`If I finish work early today, we can go to the beach.'
\end{exe}

\begin{exe}
\ex\label{ex:hoff:Imperativeexamples} \textbf{Imperatives:} \\ 
	\ex
	\begin{xlist}
\ex Argentine Spanish\\No \textbf{toqués} \textbf{[vos]} (cf. \textbf{toques} \textbf{[tú]}) eso! 
\ex Brazilian Portuguese\\Não \textbf{toc} \textbf{[PI]} (cf. \textbf{toque} \textbf{[present subjunctive]}) nisso!
\end{xlist}
`Don’t touch that!'
\begin{xlist}
\ex Brazilian Portuguese\\\textbf{Come} \textbf{[PI]} (cf. \textbf{coma} \textbf{[present subjunctive}]) isso!
\end{xlist}
`Eat that!'
\end{exe}

We intend the term ``non-standard form'' to mean simply the form not typically recognized in grammatical descriptions, which here is the potentially settledness-marking form. This is the present indicative (PI) in future-framed adverbials and conditionals as well as in Brazilian Portuguese imperatives; in Argentine Spanish \textsc{2sg} negative imperatives, however, the settledness-marking forms are those belonging to the \textit{vos} paradigm. The Real Academia Española’s Diccionario Panhispánico de Dudas (\citeyear{RAE2005}) makes contradictory statements about the acceptance of \textit{vos} negative imperatives in Argentine/Rioplatense Spanish, and we make no claims about the sociolinguistic status of these forms.
 
The remainder of the paper is organized as follows. In \sectref{sec:hoff:Settledness} we introduce the semantic-pragmatic notions of settledness and presumed settledness and anticipate their application to tense/mood variability in several syntactic contexts. In \sectref{sec:hoff:ffas} we pair qualitative and quantitative data to analyze the first of these contexts, future-framed adverbial clauses in Italian, Argentine Spanish, Brazilian Portuguese, and (Hexagonal) French. In \sectref{sec:hoff:ffcs} we examine future-referring conditionals in Italian and Brazilian Portuguese, and in section  \sectref{sec:hoff:Imperatives} we re-frame recent analyses on Argentine Spanish and Brazilian Portuguese imperatives to demonstrate that an account that considers presumed settledness is more suitable. Finally, in \sectref{sec:hoff:Advantages} we summarize the primary advantages and contributions of our analysis, both within and beyond Romance, and conclude with some promising future directions.

\section{Settledness}\label{sec:hoff:Settledness}

For our analysis, we appeal to formal semantics and to the notion of settledness (cf. \citealt{Prior1967,ThomasonGupta1980}), which has been utilized most widely in the analysis of conditionals. First, we define and exemplify contrasts in settledness proper, before distinguishing it from presumed settledness, the notion to be used here.\footnote{For the sake of brevity, we present the intuitions of settledness and presumed settledness with minimal formal semantic terminology. For a more complete explanation of the concepts used, see \citet{Hoff2019}.}

Settledness is defined as metaphysical necessity with respect to historical alternatives, and contrasts in settledness capture the fundamental difference between a fixed past and an open future. Consider the English conditionals in (\ref{ex:hoff:ifhesubmitted}) and (\ref{ex:hoff:ifhesubmits}):

\begin{exe}
\ex
\begin{xlist}
	\ex\label{ex:hoff:ifhesubmitted} If he submitted his paper to a journal, we won’t include it in our book.
	\ex\label{ex:hoff:ifhesubmits} If he submits his paper to a journal, we won’t include it in our book.
\end{xlist}
	\citep[1]{Kaufmann2002}
\end{exe}

As \citet{Kaufmann2002} explains, in (\ref{ex:hoff:ifhesubmitted}) the speaker lacks information regarding whether the paper was submitted to a journal, yet this fact is metaphysically or objectively fixed by virtue of having occurred in the past, and the outcome is thus settled.\footnote{For more on objective or metaphysical settledness, see e.g. \citet{Condoravdi2002} and \citet{Mari2013}.} In (\ref{ex:hoff:ifhesubmits}), however, the outcome is not fixed at speech time as the man in question may or may not submit his paper to a journal in the future; such cases are non-settled, because the eventuality in question has not and may never occur. 

Presumed settledness on the other hand is concerned only with eventualities which are unrealized at speech time (as in \ref{ex:hoff:ifhesubmits} above).\footnote{Since all of the morphosyntactic contrasts of interest here involve the unrealized future, presumed settledness is the notion we are concerned with, and we use the terms ``settledness'' and ``presumed settledness'' interchangeably from here on for convenience. Furthermore, as the formalism in \ref{ex:hoff:presumedsettledformal} indicates, we use ``presumed settledness'' or ``presumption of settledness'' to refer to the belief that a future eventuality will be instantiated, not merely that the question is settled one way or the other (cf. \citeauthor{Kaufmann2002} \citeyear{Kaufmann2002}, \citeyear{Kaufmann2005}).} Assuming a Kratzerian possible worlds semantics (\citealt{Kratzer1981}) and the framework of branching time, presumed settledness is the notion that every future world compatible with what the speaker believes to be the case at speech time is one in which the eventuality in question necessarily holds.\footnote{It is important to note that presumed settledness is concerned with speaker belief, rather than knowledge, since we are interested in the unknowable future. Our use of the terms belief and doxastic are intentional and contrast with \citeauthor{Johnson2016}’s (\citeyear{Johnson2016}) treatment of imperatives as conditioned by epistemicity, a point to which we return in \sectref{sec:hoff:Imperatives}.} In other words, the speaker presents a future eventuality as guaranteed in light of her beliefs. This intuition is expressed more formally in (\ref{ex:hoff:presumedsettledformal}) -- a proposition \textit{p} is presumed settled wrt a world \textit{w}, time \textit{t} (speech time), and agent \textit{a} (the speaker) iff every world $w'$ in the doxastic alternatives of \textit{w} for \textit{a} at \textit{t} is such that each of its historical alternatives $w''$ evolves into a \textit{p} world.

\begin{exe}
\ex\label{ex:hoff:presumedsettledformal} PRESUMED SETTLED(\textit{p,t,w,a}) iff $\forall w \prime$ $\in$ \textsc{DOX}$_{t}^{a}(w)$ : $\forall w \prime\prime$ $\in$ \textsc{HIST}$_{t}(w\prime)$ : $p(w\prime\prime)$
\end{exe}

On the other hand, as shown in (\ref{ex:hoff:presumednonsettledformal}), a proposition \textit{p} is presumed non-settled iff for every world $w'$ in the doxastic alternatives of \textit{w} for \textit{a} at \textit{t}, the historical alternatives $w''$ of $w'$ include both \textit{p} and not-\textit{p} worlds.

\begin{exe}
\ex\label{ex:hoff:presumednonsettledformal} PRESUMED NON-SETTLED(\textit{p,t,w,a}) iff \\
$\forall w \prime$ $\in$ \textsc{DOX}$_{t}^{a}(w)$ : $\exists w \prime$, $w\prime\prime$ $\in$ \textsc{HIST}$_{t}(w\prime)$ : $p(w\prime)$ $\wedge$ $\neg p(w\prime\prime)$
\end{exe}
	
In essence, presumed settledness is concerned with speaker confidence about the way the future will unfold. Although the presumption of settledness is binary (either all future worlds are \textit{p} worlds, or not-\textit{p} worlds are also permitted), this determination is made based on gradient ``component parts” from which speakers derive their confidence about the future. For example, speakers tend to have greater confidence in the realization of immediate eventualities (e.g. in 5 minutes) than distant ones (e.g. in 10 years), and the temporal distance between speech time and expected realization of a future eventuality is of course gradient (see \citet{Condoravdi2002} for discussion on how future possibilities cease to be ``live options'' as time progresses in a branching time framework). Similarly, eventualities with specific moments of expected realization (e.g. at 10:03pm) are more likely to be presumed settled by speakers than are non-specific or open-ended eventualities (e.g. some day), and this specificity is also gradient.

Of course, these contextual factors that contribute to speaker confidence are often interconnected in practice, despite being theoretically separable. For example, we can imagine a context where a future eventuality is expected to occur in the near future but at an unspecified time, or in the distant future but at a highly specific time. We can even conceive of contexts where an eventuality is uncertain to occur at all, but if it does, it must happen at a precise moment years after speech time. Thus, though immediacy is perhaps the most commonly cited factor in linguistic discussions of future reference, it is neither necessary nor sufficient and is instead best conceived as a component of speaker confidence (a point that proves to be important in our analysis of imperatives in \sectref{sec:hoff:Imperatives}). It is precisely for this reason that in \citeauthor{Hoff2019}’s (\citeyear{Hoff2019}, \citeyear{Hoff2020}, \citeyear{HoffForthcoming}) experimental questionnaire analyses of \textsc{ffa}s, presumed settledness is operationalized in terms of 3 contextual indicators of speaker confidence: explicit expressions of certainty, immediacy, and temporal specificity. In order to most clearly examine the effects of presumed settledness on form choice and acceptability in a quantitative analysis, Hoff’s stimuli were designed to be maximally distinct in terms of these three interrelated contextual factors (either [+certain,+immediate,+specific] or [-certain,-immediate,-specific]). However, as explained in \citet{Hoff2019}, additional factors may also color speakers’ beliefs about future eventualities. For example, speakers’ desires for the future, their assumptions about what is typical, and their expectations that habitual situations will continue into the future may all contribute to speaker confidence. In general terms, then, we can say that a speaker's degree of certainty that an event will occur at all is primary and that any additional elements related to settledness that further condition use of the settledness-marking form are contextually determined.

Before moving on to the data, it is important to emphasize that presumed settledness is a single pragmatic notion related to speaker confidence. Even though the sources of this confidence can be numerous, varied, and highly contextual as explained above, the potential to express settledness allows a speaker to convey a straightforward message to her interlocutor, namely that she believes that the future eventuality in question is guaranteed to occur. To highlight the desirability of this very simple notion, consider the laundry list of meaning differences in (\ref{ex:hoff:laundrylist}), which have been attributed in the linguistic literature to the contrast between the futurate present and the periphrastic future in French.

\begin{exe}
\ex\label{ex:hoff:laundrylist} immediacy, imminence, guarantee of truth, already initiated eventuality, speaker confidence about outcome, action is determined/planned, necessary conditions already fulfilled, unavoidable, speaker involvement\\
(adapted from \citealt{PoplackTurpin1999})
\end{exe}

\noindent Most, if not all, of these purported meaning differences are united under the umbrella notion of presumed settledness, thus allowing for an analysis that unifies, rather than invents distinctions between, similar pragmatic conditions. 

Still, it is essential to remember that contrasts based on presumed settledness are probabilistic, as in other cases of morphosyntactic variation. As we will demonstrate in the sections to follow, the degree of variability differs appreciably from one syntactic environment to another and from language to language. Put another way, the form-function isomorphism between a particular morphosyntactic variant and settledness is not equivalent in all contexts or all languages. Distinctions are more clear-cut in some cases, where the settledness-marking form is virtually impossible except in the most certain contexts (e.g. Spanish \textsc{ffa}s), than in others, where the pragmatic restrictions on this form have become somewhat weakened. Italian in particular demonstrates this relaxing of restrictions, as we will show. Additionally, in all cases the standard form remains an acceptable option, even to express future eventualities that speakers would almost certainly presume to be settled.\footnote{This does not mean, however, that the standard form is always the preferred option. Indeed, in some cases, the settledness-marking form is preferred due to its greater pragmatic informativeness.} Nevertheless, the key fact remains that in each morphosyntactic contrast analyzed here, the settledness-marking form is used and accepted most where the context clearly indicates high speaker confidence.

\section{Future-framed adverbials}\label{sec:hoff:ffas}

Grammatical descriptions of several Romance varieties either suggest or explicitly state that there is only one form possible in an adverbial subordinate clause that refers to a future eventuality. In Italian and Hexagonal French the prescribed form is the future indicative, in Argentine Spanish it is the present subjunctive, and in Brazilian Portuguese it is the future subjunctive. However, \citeauthor{Hoff2019} (\citeyear{Hoff2019}, \citeyear{Hoff2020}, \citeyear{HoffForthcoming}) shows that this description is empirically inaccurate in that it ignores the possibility of the present indicative in all four languages, as in (\ref{ex:hoff:ffaexamplesffasec}).\footnote{Whereas tense/mood variation in future-referring main clauses is more widely studied (see e.g. \citealt{Grimm2015}, \citealt{GudmestadEdmondsDonaldsonCarmichael2018}, and \citealt{PoplackTurpin1999}), subordinate clauses have been largely ignored. See \citeauthor{Hoff2019} (\citeyear{Hoff2019}, \citeyear{Hoff2020}, \citeyear{HoffForthcoming}) for a more comprehensive discussion of the acknowledgment of this variation (or lack thereof) in grammars, linguistic analyses, and pedagogical materials.}

\begin{exe}
\ex\label{ex:hoff:ffaexamplesffasec}
a.         Ti chiamo quando \textbf{arrivo} \textbf{[PI]} (cf. \textbf{arriverò} \textbf{[future indicative]}). \\
b.         Te llamo cuando \textbf{llego} \textbf{[PI]} (cf. \textbf{llegue} \textbf{[present subjunctive]}). \\
c.         Te ligo quando \textbf{chego} \textbf{[PI]} (cf. \textbf{chegar} \textbf{[future subjunctive]}). \\
d.         Je t’appelle quand j’\textbf{arrive} \textbf{[PI]} (cf. \textbf{arriverai} \textbf{[future indicative]}).\footnote{In French, the periphrastic future is a possible third variant (e.g. \textit{Je vais t’appeler quand je vais arriver}). However, Hoff’s (\citeyear{HoffForthcoming}) questionnaire data from Hexagonal French show that this is by far the least productive variant in \textsc{ffa}s.} \\
`I’ll call you when I arrive.'
\end{exe}
                        
Hoff (\citeyear{Hoff2019},\citeyear{Hoff2020}, \citeyear{HoffForthcoming}) provides both quantitative and qualitative data to show that in each of these languages, the variation between the standard form and PI is conditioned by presumed settledness. That is, PI is used most when contextual clues indicate that the speaker perceives the future eventuality in the subordinate clause to be guaranteed, as in the questionnaire stimulus in (\ref{ex:hoff:grandma}), whereas the standard form is preferred when the eventuality is presumed non-settled as in (\ref{ex:hoff:graduate}) (though as mentioned in \sectref{sec:hoff:Settledness} the standard form is also acceptable in settled contexts).

\begin{exe}
\ex
\begin{xlist}
\ex\label{ex:hoff:grandma}
You’re driving to your grandma’s house and you know that your mom must be worried, as always. You call her and tell her that you’re 5 minutes away. You say: When I arrive, I’ll let you know. 
\ex\label{ex:hoff:graduate}
You’ve been working and going to school part time for years, and you’re not sure when you’ll be able to graduate. You still have several courses left and work takes up all your free time. You say: When I finish school, I’ll look for a better job.
\end{xlist}
(translations of Spanish stimuli from \citealt{Hoff2019})
\end{exe}

This pattern is made clear in \tabref{tab:hoff:rates}, which compares rates of choice of PI by Argentine, Italian, and French participants in \citeauthor{Hoff2019}’s (\citeyear{Hoff2019}, \citeyear{Hoff2020}, \citeyear{HoffForthcoming}) forced-choice questionnaires.\footnote{\citeauthor{Hoff2019} (\citeyear{Hoff2019}, \citeyear{Hoff2020}) included two versions of the forced-choice Italian questionnaire, one where the main-clause verb always appeared in the present and another with all main-clause verbs in future tense. All results presented here come from the present tense version, but see \citet{Hoff2020} for more on the role of tense-matching.}

\begin{table}
\begin{tabular}{lrrr}
\lsptoprule
& \textbf{Argentine Spanish} & \textbf{Italian} & \textbf{Hexagonal French} \\
\midrule
\textbf{Settled} & 37\% & 86\% & 69\% \\
& (221/604) & (741/864) & (760/1096) \\
\textbf{Non-settled} & 3\% & 45\% &18\%  \\
& (19/604) & (33/864) & (196/1096) \\
\lspbottomrule
\caption{Rate of choice of PI in settled vs. non-settled contexts across languages}
\label{tab:hoff:rates}
\end{tabular}
\end{table}

Despite obvious differences in PI selection across these languages, the pattern whereby PI is preferred in settled, rather than non-settled contexts, is clear in each. Even in Italian, where choice of PI was fairly high for both settledness conditions, no participant (0/108) chose PI more often in non-settled than in settled contexts; indeed, an impressive 92\% (99/108) of participants chose PI for all settled stimuli. Furthermore, mixed-effects logistic regression analyses confirm that settledness condition has a significant effect on PI choice in all three languages. These findings are echoed for Argentine Spanish and Italian in \citeauthor{Hoff2019}’s (\citeyear{Hoff2019}, \citeyear{Hoff2020}) acceptability judgment data, which reveal that PI is rated significantly more positively in settled than in non-settled contexts.\footnote{\citeauthor{HoffForthcoming}’s (\citeyear{HoffForthcoming}) Hexagonal French questionnaire contained only a forced-choice task, thus acceptability judgment data are not yet available for this variety.}

While \citeauthor{Hoff2019}’s (\citeyear{Hoff2019}, \citeyear{Hoff2020}) quantitative analyses address only \textit{when} clauses, his qualitative data suggest that \textsc{ffa}s containing other adverbs (\textit{as soon as, until, after}, etc.) work quite similarly. The following tweets containing Brazilian Portuguese \textit{assim que}, Argentine Spanish \textit{apenas}, and Italian \textit{(non) appena} (`as soon as') exhibit this:

\begin{exe}
\ex\label{ex:hoff:assimqueacordo} Indo dormir, amanhã assim que \textbf{acordo} \textbf{[PI]} volto. \\
`Going to bed. Tomorrow as soon as I wake up, I’ll be back.'
\end{exe}

\begin{exe}
\ex\label{ex:hoff:apenasterminas} Apenas \textbf{terminás} \textbf{[PI]} de trabajar, necesito urgente la Sube. \\
`As soon as you finish work, I need the subway pass right away.'
\end{exe}

\begin{exe}
\ex\label{ex:hoff:appenaposso} Io devo andare a scuola, appena \textbf{posso} \textbf{[PI]} ricomincio a twittare. Appena \textbf{torno} \textbf{[PI]} voglio vedere questo hashtag in tendenza \#ItalyWantsAnotherConcert. \\
`I have to go to school. As soon as I can, I’ll start tweeting again. As soon as I get back, I want to see this hashtag trending.'
\end{exe}

In fact, preliminary exploration of these adverbs suggests that those conveying a stronger presumption on the part of the speaker that the eventuality will occur lend themselves even more readily to PI, though quantitative confirmation of this hypothesis is still needed. Examples (\ref{ex:hoff:assimqueacordo}-\ref{ex:hoff:appenaposso}) provide evidence of additional contextual factors that condition PI use, namely speakers’ common assumptions about what is typical and speakers’ expectations that recurring eventualities in the present will continue into the future. In (\ref{ex:hoff:assimqueacordo}), for example, the speaker appears to presume settled that he will wake up the next morning, since he has done so every day of his life and given the generally shared assumption that we will not die in our sleep. Similarly, (\ref{ex:hoff:apenasterminas}) and (\ref{ex:hoff:appenaposso}) reflect speakers' assumptions that working or being at school are daily activities that typically operate on a regular and predictable schedule and that we can confidently plan our actions around this.\footnote{For more on schedules and plans and their central importance for the expression of futurity more generally, see \citeauthor{Copley2002} (\citeyear{Copley2002}, \citeyear{Copley2008}). See \citet{Copley2018} for  related discussion on the role of intentions and disposition in futurate use.}

While Hoff’s research (\citeyear{Hoff2019}, \citeyear{Hoff2020}, \citeyear{HoffForthcoming}) constitutes the first applications of presumed settledness to tense/mood variation in \textsc{ffa}s across Romance, in this paper we elaborate on those findings and extend them to other syntactic environments, as well as to other varieties. The tweets below, for example, show that certain other varieties of Spanish also allow PI for settled future eventualities (\ref{ex:hoff:cuandosalgo}-\ref{ex:hoff:apenasllego}), as does Canadian French (\ref{ex:hoff:desquej'arrive}-\ref{ex:hoff:quandjefinis}). 

\begin{exe}
\ex\label{ex:hoff:cuandosalgo} Te escribo cuando \textbf{salgo} de clases a ver si sigues ahí, yo salgo a las 5:20. (Lima)\\
`I’ll write you when I get out of class to see if you’re still there. I get out at 5:20.'
\end{exe}

\begin{exe}
\ex\label{ex:hoff:apenasllego} Ahora ando en la pega, apenas \textbf{llego} te agrego. (Santiago de Chile)\\
`I’m at work right now. As soon as I get home, I’ll add you.'
\end{exe}

\begin{exe}
\ex\label{ex:hoff:desquej'arrive} On est encore au resto j’arrive bientôt! Je t’appelle dès que j’\textbf{arrive}! (Montreal)\\
`We’re still at the restaurant, I’ll be there soon. I’ll call you as soon as I arrive.'
\end{exe}

\begin{exe}
\ex\label{ex:hoff:quandjefinis} Je viens te rejoindre quand je \textbf{finis} de travailler. (Québec City)\\
`I’ll come join you when I finish work.'
\end{exe}

Before moving on to conditionals with future reference, it is important to recognize their appreciable overlap with the \textsc{ffa}s presented here. It is traditionally claimed that \textit{when}-clauses presuppose the eventuality described, while \textit{if}-clauses do not; however, this view is overly simplistic. First, the data from \citeauthor{Hoff2019} (\citeyear{Hoff2019}, \citeyear{Hoff2020}) clearly show that \textit{when}-clauses are cross-linguistically used for both near-certain and far less certain future eventualities and that tense/mood selection of the subordinated verb probabilistically maps onto these degrees of speaker confidence. Second, \citet{Schwenter1999} demonstrates that the hypotheticality of \textit{if}-clauses is not entailed, but rather conversationally implicated, a point discussed further in the next section. Thus, given the overlap between \textit{when}- and \textit{if}-clauses, it is to be expected that future-referring subordinate clauses with \textit{if} and \textit{when} would allow similar morphosyntactic contrasts that enable speakers to encode their beliefs about the future. 

\section{Future-framed conditionals}\label{sec:hoff:ffcs}

\citeauthor{Kaufmann2002}’s (\citeyear{Kaufmann2002}, \citeyear{Kaufmann2005}) work on settledness and presumed settledness began with English conditionals of different sorts, such as those exemplified in (\ref{ex:hoff:ifhesubmitted}-\ref{ex:hoff:ifhesubmits}) above. However, for our purposes, we will only be examining future-framed conditional sentences, i.e. those whose protases refer to verbal eventualities located posterior to speech time. We concentrate on two Romance languages that allow for morphosyntactic variation in such contexts: Brazilian Portuguese and Italian.

In Brazilian Portuguese, \textsc{ffc}s typically, and normatively, occur with the future subjunctive form in the protasis. Thus, in (\ref{ex:hoff:sechegarchego-a}), the future subjunctive form \textit{chegar} situates the eventuality of arriving at a point that is posterior to speech time. However, this form competes with the present indicative (\textit{chego}) as in (\ref{ex:hoff:sechegarchego-b}); both are translatable in the same way into a language like English where no such distinction is found. 

\begin{exe}
\ex\label{ex:hoff:sechegarchego} 
\begin{xlist}
\ex\label{ex:hoff:sechegarchego-a} Se \textbf{chegar} \textbf{[FS]} na hora, te ligo.
\ex\label{ex:hoff:sechegarchego-b} Se \textbf{chego} \textbf{[PI]} na hora, te ligo.
\end{xlist}
`If I arrive on time, I'll call you.'
\end{exe}

The difference between the two forms concerns presumed settledness: the present indicative in (\ref{ex:hoff:sechegarchego-b}) is preferred when the speaker is confident in an on-time arrival, while (\ref{ex:hoff:sechegarchego-a}) is neutral in that regard. Grammars of Portuguese rarely, if ever, mention this variation, and typically relegate the present indicative to protases with present temporal reference (often termed ``factual'' or ``real'' conditionals; see \citealt{Lobo2013}). \citeauthor{Gomes2008} (\citeyear{Gomes2008}: 224) is one of the few authors to acknowledge this variation, stating that ``some dialects of Portuguese would use [the indicative]'' in future-framed conditionals, and giving the example \textit{Se ele está cansada, é melhor parar} `If she is tired, it’s better to stop’ where the protasis verb \textit{está} is in the present indicative instead of the future subjunctive \textit{estiver}. In this example, the temporal reference of the protasis could be compatible with the future (paraphrasable as ``if she is tired, and I believe she will be''). Nevertheless, \citet{Gomes2008} does not mention which dialects of Portuguese he is talking about, and we have found no mention of any such dialectal differences in the literature or indeed any speakers who recognize such a difference. We believe, therefore, that such a difference does not actually exist; instead, the possibility of variation in future-framed protases is due to presumed settledness, and to speakers’ strategic choices for expressing their confidence in the eventuality described in the protasis. 

When the context of the future-framed conditional sentence is already biased toward the settledness of the eventuality in question, then the present indicative is actually the most natural choice in Brazilian Portuguese. For instance, consider a context where a young boy is known by all interlocutors present to eat beans every day at lunch. In this case, it would be normal to use the present indicative in the protasis:

\begin{exe}
\ex\label{ex:hoff:seelecome} Se ele \textbf{come} \textbf{[PI]} feijão amanhã no almoço, ninguém vai se surpreender (cf. \textbf{comer} \textbf{[future subjunctive]}). \\
`If he eats beans at lunch tomorrow, nobody will be surprised.'
\end{exe}

Such a context could be further strengthened toward the bean-eating with the addition of an adverbial phrase at the end of the protasis, e.g. \textit{como ele sempre faz} `as he always does', thereby making even more explicit the speaker’s confidence that the boy will eat beans again tomorrow.

Whereas in Portuguese the standard form in \textsc{ffc}s is the future subjunctive, in Italian, the future indicative is prescribed. The pragmatic information that may be conveyed by choice of PI, however, remains the same. Consider (\ref{ex:hoff:sefa}) for instance.

\begin{exe}
\ex\label{ex:hoff:sefa} Se \textbf{fa} [PI] bel tempo domani, gioco a tennis (cf. \textbf{farà} \textbf{[future indicative]}).\\
`If the weather is nice tomorrow, I’ll play tennis.'
\end{exe}

A speaker is more likely to choose PI here when she presumes settled that the weather will indeed be nice the next day and, as a result, that she will play tennis. A similar intuition is observed in Aski and Musumeci’s (\citeyear{AskiMusumeci2014}: 267) advice to learners of Italian, who suggest that PI can be used in conditionals ``referring to the present time, the near future, or general truths.'' Either settledness proper or presumed settledness account for all three. However, as alluded to at the end of \sectref{sec:hoff:Settledness}, Italian allows for more liberal use of PI than does Brazilian Portuguese (see also Italian’s relatively high rate of PI in non-settled \textsc{ffa}s in \tabref{tab:hoff:rates}). Thus, even though PI’s primary domain of use is in presumed settled contexts and speakers are sensitive to this pragmatic meaning in cases of more explicit contrast with the future indicative, PI may also appear outside of this domain. Though it can be hypothesized that PI was once more strictly limited to settled contexts and has since undergone a relaxing of constraints, we leave the diachronic verification of this possibility for future research.

In other Romance languages such as Spanish and French, as well as in English, there is no contrast between two forms in the protasis of conditional sentences (even though there may have been in diachrony): the present indicative is the only form used for future-framed expression. Does this mean that presumed settledness is irrelevant to this domain of the grammar in these languages? Of course not. Speakers can convey presumed settledness via other means (e.g. adverbial expressions that convey confidence) or they can simply allow the context to guide interpretation. This is a normal situation cross-linguistically: what is grammaticalized in one language is conveyed via lexical and/or pragmatic means in another (evidentiality is a prime example of a meaning domain where this is true). Grammars, as well as semanticists, tend not to recognize this possibility because of their assumption that the encoded meaning of conditional conjunctions like \textit{se/si/if} is one of hypotheticality; however, there is clear evidence that their hypotheticality is merely a (generalized) conversational implicature, and therefore is cancelable by accompanying contextual information (\citealt{Schwenter1999}; \citealt{Levinson2000}). Adopting this view allows one to explain why there is not only an interpretational overlap between \textit{if} and \textit{when} clauses, but also why in some languages one form can be used for both meanings (e.g. German \textit{wenn}). Note also that in English, despite a lack of verbal morphology to express presumed settledness in conditional sentences, there exists a common grammaticalized expression for this purpose: \textit{if and when}. Thus, we can say \textit{If and when I get there, I’ll call you}, but not \# \textit{If or when I get there, I’ll call you}. This strengthening of \textit{if} is explicable under our account, since it allows a speaker to express increased confidence in the realization of the eventuality in question, i.e. to convey that the situation is presumed settled.

\section{Imperatives}\label{sec:hoff:Imperatives}

In this section, we first address negative imperatives in Argentine Spanish, followed by both affirmative and negative imperatives in Brazilian Portuguese. In both languages, presumed settledness helps regulate the choice of imperative form in context.

\citet{Johnson2013} offers a quantitative analysis of \textsc{2sg} negative imperatives in Argentine Spanish, where there is variation between forms belonging to the \textit{tú} (\ref{ex:hoff:nocomas-a}) and \textit{vos} (\ref{ex:hoff:nocomas-b}) paradigms.

\begin{exe}
\ex\label{ex:hoff:nocomas}
\begin{xlist}
\ex \label{ex:hoff:nocomas-a} ¡No \textbf{comas} [tú] eso! 
\ex \label{ex:hoff:nocomas-b} ¡No \textbf{comás} [vos] eso!\\
`Don’t eat that!'
\end{xlist}
\end{exe}

In her elicitation task, \citeauthor{Johnson2013}’s (\citeyear{Johnson2013}) participants were presented with written contexts and asked to select the form they would use: a \textit{tú} imperative, a \textit{vos} imperative, or either. Contexts were controlled for immediacy such that in so-called ``urgent'' contexts, the action (or a preparatory stage of it) was already in progress (\ref{ex:hoff:butcher}), whereas in ``neutral'' contexts the action had not yet been performed (\ref{ex:hoff:cheese}). Johnson found a significant effect (p<.01) for immediacy such that \textit{vos} forms were favored in immediate contexts (\ref{ex:hoff:butcher}), but disfavored in neutral ones (\ref{ex:hoff:cheese}).
  
\begin{exe}
\ex
\begin{xlist}
\ex\label{ex:hoff:butcher} You’re at the butcher shop and you ask for a kilo of flank steak for a barbecue. The butcher starts cutting you milanesas. You only asked for flank steak.
\ex\label{ex:hoff:cheese} You’re at the supermarket buying a kilo of cheese. You want it whole, not cut, and you have to tell this to the boy who works there.
\end{xlist}
(adapted from \citealt[168]{Johnson2013})
\end{exe}

In a second task, participants were presented with brief contexts followed by four utterance types: a \textit{tú} negative imperative, a \textit{vos} negative imperative, a yes/no question, and a declarative. Participants then used a 5-point scale to evaluate each utterance type for how certain the speaker was that her interlocutor would have performed the action described. Johnson’s results showed that \textit{vos} forms were evaluated as conveying significantly more epistemic certainty toward interlocutors' performing the action than their \textit{tú}-paradigm counterparts (p<.01); whereas \textit{tú} negative imperatives received a mean certainty score of 2.3 out of 5, \textit{vos} negative imperatives received a mean score of 3.81, indicating (according to her terminology) a stronger epistemic bias.

Thus, \citet{Johnson2013} concludes, the use of \textit{vos} forms conveys pragmatic information about the speaker’s ``epistemic commitment'' to the addressee’s future intentions; in other words, \textit{vos} negative imperatives implicate not only `don’t do that,' but also `I believe you were going to do that' (151). Although we avoid the term “epistemic commitment” to describe speaker’s beliefs about the future, the meaning \citet{Johnson2013} attributes to \textit{vos} negative imperatives essentially amounts to a presumption of settledness. In saying \textit{No me pong\'{a}s} [vos] \textit{vainilla} `Don’t give me vanilla' to an ice cream shop employee, the speaker expresses that according to her beliefs, it was guaranteed that she was going to receive vanilla ice cream or, put more precisely, that all possible unfoldings of the future would be vanilla-receiving worlds. The notion of settledness, then, provides a clearer connection between certain of Johnson’s ``urgent'' contexts where the speaker sees the undesired action already taking place (e.g. beginning to scoop the wrong flavor, or to cut the wrong meat as in \ref{ex:hoff:butcher} above) and those in which the action has not yet been performed, but the speaker has a high degree of confidence about her interlocutor’s future behavior. In the first case the undesired action is objectively settled by virtue of already being realized, whereas in the second it is presumed settled by the speaker. Therefore, treating the \textit{t\'{u}/vos} contrast as conditioned by a single pragmatic notion, rather than two distinct factors analyzed through separate tasks, provides a clearer picture of the relationship between immediacy and certainty.

An extension of Johnson’s work has been made more recently by \citeauthor{LambertiSchwenter2015} (\citeyear{LambertiSchwenter2015}, \citeyear{LambertiSchwenter2018}). Unlike the case of Argentine Spanish, where the morphological contrast between imperative forms is only found in the negative paradigm, in Brazilian Portuguese there is a more generalized contrast in both affirmative and negative \textsc{2sg} imperatives. This contrast is shown in (\ref{ex:hoff:fechafeche}) and (\ref{ex:hoff:jogajogue}) below, where the (a) sentences illustrate the historical \textsc{2sg} imperative form (identical to the \textsc{3sg} present indicative) associated with the subject pronoun \textit{tu}, and the (b) versions the imperative form that is identical to the \textsc{3sg} present subjunctive conjugation, which is associated with the more commonly used \textsc{2sg} subject pronoun \textit{você} (derived historically from the \textsc{3sg} NP \textit{vossa merced} ‘your mercy’ which then grammaticalized as a \textsc{2sg} form).

\begin{exe}
\ex\label{ex:hoff:fechafeche}
\begin{xlist}
\ex 	\textbf{Fecha} \textbf{[present indicative]} a porta! 
\ex \textbf{Feche} \textbf{[present subjunctive]} a porta!
\end{xlist}
`Shut the door!'
\end{exe}

\begin{exe}
\ex\label{ex:hoff:jogajogue}
\begin{xlist}
\ex Não \textbf{joga} \textbf{[present indicative]} isso no lixo!
\ex Não \textbf{jogue} \textbf{[present subjunctive]} isso no lixo!
\end{xlist}
`Don’t throw that away!'
\end{exe}
	
Brazilian Portuguese speakers show considerable variation between these two forms, and regionally there are distinct tendencies between them, e.g. there is much more use of the indicative (a) forms in the south of Brazil, while the subjunctive (b) forms are more prevalent in northeastern regions of the country. However, there are clear patterns that emerge regardless of dialect; these patterns can all be understood under the umbrella of presumed settledness. 
\citet{LambertiSchwenter2018} used an online survey administered to native speakers of Brazilian Portuguese. In their instrument, the authors paired nine frequent verbs with different adverbial expressions which differed in temporal immediacy, ranging from the here-and-now (\textit{agora} `now' or \textit{daqui a uma hora} `an hour from now') to maximally distant or nonexistent timeframes (\textit{sempre} `always' or \textit{nunca} `never'). These pairings resulted in seven ``immediate'' contexts and three ``non-immediate'' contexts. Their findings revealed that in all four of the BP dialects examined from different regions of Brazil, there was significantly more indicative chosen in the ``immediate'' contexts (e.g. 57.3\% in S\~{a}o Paulo), yet the indicative was chosen significantly less in the ``non-immediate'' contexts (e.g. only 3.1\% in S\~{a}o Paulo). While Lamberti and Schwenter focused on the temporal dimension of the commands, it is important to point out that temporality alone cannot account for these findings. Thus, in the case of adverbs like \textit{sempre} and \textit{nunca}, which quantify over unbounded time periods and are also typically applicable to all situations/interlocutors, there is no assumption of presumed settledness such that the speaker believes that the addressee will or will not perform the verbal action. In the case of the ``immediate'' adverbs employed by the authors, there is greater orientation to the here-and-now, and also to a specific interlocutor, thus resulting in a relatively greater presumption of settledness. 

How then does presumed settledness work in the context of imperatives, whether affirmative or negative? In the case of affirmative imperatives, there is often an assumption that the addressee will not carry out the action depicted by the verb unless otherwise commanded to do so. A speaker’s choice of the indicative in such a context reflects the speaker’s assessment that the addressee objects to carrying out said action, i.e. the objection is presumed settled without the issuing of the imperative. When a speaker instead chooses the subjunctive to encode the imperative, the speaker recognizes that their interlocutor may still choose to carry out the action described, i.e. it is not presumed settled that the addressee has refused a priori. For this reason, it is easy to comprehend why \citet{LambertiSchwenter2018} found that BP speakers choose the indicative imperative for ``here-and-now'' commands (i.e. prototypical commands per \citealt{Aikhenvald2010}) where the speaker is more aware of (a specific) addressee’s opposition, and the subjunctive imperative for displaced commands where the addressee (including generic addressees) has yet to oppose or even opine regarding the verbal action in question.

For negative commands, the distinction between ``cessative'' and ``preventive'' imperatives (\citealt{Haverkate1979}) is relevant to the choice of imperative form in BP. Cessative imperatives command addressees to stop some action or behavior that they are already engaged in. It is easy to see how presumed settledness would apply here: given that the addressee is already doing something undesirable from the speaker’s perspective, the speaker chooses the form that best matches this situation, which in BP is the indicative. On the other hand, a preventive imperative is one whose purpose is to prospectively keep an addressee from carrying out the verbal action in question. Thus, in this case, it is typically not presumed settled that the addressee has plans to carry out the action, and therefore the BP speaker has a strong tendency to choose the subjunctive form for the imperative.

Finally, it is noteworthy that in the case of \textit{tú/vos} negative imperative alternation, the contrast is not one of tense or mood, as in the \textsc{ffa}s and \textsc{ffc}s analyzed in sections \sectref{sec:hoff:ffas} and \sectref{sec:hoff:ffcs} or in the Brazilian Portuguese imperatives discussed here, but rather of distinct \textsc{2sg} verbal person paradigms. Thus, although Argentine Spanish negative imperatives also fall under the conceptual umbrella of future reference and speaker confidence in future eventualities, they are unique in that they do not rely on the morphological contrasts most typically used to convey temporality or speaker (lack of) belief, such as tense or mood. It is essential, therefore, that future research be open to this possibility in other languages, in order to maximize the cross-linguistic utility of settledness.

\section{Advantages \& contributions}\label{sec:hoff:Advantages}

Presumed settledness is an intuitive and economical notion with immense theoretical potential for describing morphosyntactic variation. Its application is cross-linguistic, rather than language-specific, as our analysis of Spanish, Portuguese, French, and Italian demonstrates. It is cross-constructional rather than specific to a single syntactic environment, and is a central consideration in accounting for form choice in multiple morphosyntactic phenomena with future temporal reference. Furthermore, an analysis based on presumed settledness brings together notions such as certainty, immediacy, specificity, and a host of others, the interconnectedness of which has not been fully recognized in previous research. Even though speaker confidence is multifaceted, the meaning speakers convey by choosing a settledness-marking form (particularly in the languages and syntactic environments where PI is most restricted and thus most informative) is clear. It allows them to express confidence about future plans, to reinforce promises, and to portray as guaranteed their desired unfolding of the future, which is anything but guaranteeable (hence our insistent references here to speaker belief rather than knowledge/epistemicity). This understanding of what speakers gain by utilizing such a contrast thus enables us to make predictions about the contexts in which speakers will most readily use and accept non-standard forms. Even in languages such as Italian where PI may be extending beyond settled contexts, it still remains the case that speakers use and accept PI most when the discourse context justifies a presumption of settledness regarding the future eventuality. Thus, by analyzing several future-related contexts together, we move beyond the mere pairing of pragmatic correlates with form choice (e.g. immediacy, specificity) and instead begin to explain why similar variability appears across syntactic environments and languages. An analysis based on presumed settledness highlights the intuitive nature and interactional utility of these morphosyntactic contrasts.
 
In sum, our work contributes to an understanding of grammatical variation that identifies pragmatic forces at work without expecting perfect symmetry between form and function (\citealt{PoplackDion2009}). The role of presumed settledness in the morphosyntactic contrasts presented here is therefore probabilistic, rather than absolute, and such variability is expected as in any linguistic system. This is nowhere more clear than in Italian \textsc{ffa}s and \textsc{ffc}s where, despite a clear pattern whereby PI is preferred when the context is compatible with a speaker’s presumption of settledness, appreciable inter- and intra-speaker variation is nevertheless observed.

Finally, we recognize the potential role of the standard norm in speakers’ use and the distinct power of the norm depending on the language in question, and therefore the variable acceptability of settledness-marking forms and the consequent differences in frequency with which speakers may encounter the standard and non-standard forms in daily interactions. On the one hand, the settledness-marking form is pragmatically more restricted and appears in only a subset of the contexts where the standard form is possible; on the other hand, however, it is the pragmatically more informative variant, as well as the form that allows speakers to portray themselves as confident, assured, and in control of their own futures. 

\il{Latin} %add "Latin" to language index for this page

\is{Cognition} %add "Cogntion" to subject index for this page

\section*{Acknowledgements}
We wish to thank Ashwini Deo for her expertise in semantics and Janice Aski and Terrell Morgan for helpful comments on earlier versions of this work. We also thank two anonymous reviewers and the audience members at LSRL 2019 for their invaluable feedback.

\printbibliography[heading=subbibliography,notkeyword=this]

\end{document}