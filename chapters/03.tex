\documentclass[output=paper,colorlinks,citecolor=brown]{langscibook}

\title{The voice of existential \textit{on}}
\author{
    J.-Marc Authier\affiliation{The Pennsylvania State University}
    and 
    Lisa A. Reed\affiliation{The Pennsylvania State University}
}

\abstract{
The external arguments of short passives and French existential \textit{on} constructions share strikingly similar properties: both display discourse and scopal inertness, both fail to provide an antecedent for a \textsc{pro} subject of a passive infinitival, and neither is compatible with an unaccusative verb. Such similarities have yet to be fully explained. Using Merchant’s (2013) observation that ellipsis is subject to identity between phrase markers, we argue that existential \textit{on} sentences contain a non-active Voice head that existentially binds an external argument that remains syntactically unprojected, just as Bruening (2013) argues is the case for short passives. We further argue that \textit{on} is the default agreement spellout that obtains when tensed T fails to find a goal bearing the appropriate valued ${\varphi}${}-features, assuming with Preminger (2009, 2014) that when Agree fails, the unvalued features on the probe retain their preexisting or default values. Finally, we argue that EPP in existential \textit{on} constructions is satisfied on the assumption that it reduces to An’s (\citeyear{an2007a}) Intonational Phrase Edge Generalization (IPEG) and we show that IPEG, combined with our theory of existential \textit{on}, correctly predicts the existence of a silent counterpart to existential \textit{on} in ECM contexts.
}

\IfFileExists{../localcommands.tex}{%hack to check whether this is being compiled as part of a collection or standalone
  % add all extra packages you need to load to this file

\usepackage{tabularx,multicol,multirow}
\usepackage{url}
\urlstyle{same}

\usepackage{listings}
\lstset{basicstyle=\ttfamily,tabsize=2,breaklines=true}

\usepackage{langsci-basic}
\usepackage{langsci-optional}
\usepackage{langsci-lgr}
\usepackage{langsci-gb4e}
%    \let\eachwordone=\it % Ch 14, 18

\usepackage{jambox}
\usepackage{subfigure}
\usepackage{tablefootnote}
\usepackage[nameinlink, noabbrev]{cleveref}
\crefname{enumi}{example}{examples}

\usepackage{bbding}
%\usepackage{linguex}
\usepackage{stmaryrd}

\usepackage{tipa}
\let\ipa\textipa
\usepackage{vowel}
\newcommand{\BlankCell}{}
\usepackage{ot-tableau}

\usepackage{forest}
\useforestlibrary{linguistics}
\usepackage[noeepic]{qtree}
\usepackage{pstricks, pst-xkey, pst-jtree}
\usepackage{tikz-qtree}
\usepackage{tikz-qtree-compat}
\usepackage{tree-dvips}

\usepackage{lastpage}
\usepackage{hyperref}
\usepackage{xltxtra}

\usepackage{ragged2e}
%\usepackage{subcaption}
\usepackage{floatrow}
\usepackage{float}

\usepackage[normalem]{ulem} % Pour les textes barrés
\usepackage{ifthen} 

\usepackage{todonotes}

  \newcommand*{\orcid}{}

\makeatletter
\let\theauthor\@author
\makeatother

\papernote{\scriptsize\normalfont
    \theauthor.
    \titleTemp. 
    To appear in: 
    Chad Howe and Pilar Chamorro and Timothy Gupton and Margaret Renwick.
    Theory, Data, and Practice: Selected papers from the 49th Linguistic Symposium on Romance Language
    Berlin: Language Science Press. [preliminary page numbering]
}

% Workaround for subscripts with capital letters
\newcommand{\capsub}[1]{\ensuremath{_\text{#1}}}

% Chapter 10: Table-like presentation within example environment
% classical latin > {*}late latin > old french  earlier > later   gloss
\newcommand{\montanoboxi}[7]{\parbox{2cm}{#1} > {#2}\parbox{2cm}{#3} > \parbox{1.5cm}{\textit{#4}} \parbox{1.2cm}{#5}\ > \parbox{1.2cm}{#6} \parbox{1.5cm}{#7}}
% {*}latin > earlier OF [ipa] > early OF   gloss
\newcommand{\montanoboxii}[6]{{#1}\parbox{1.9cm}{\textit{#2}} > \parbox{1.3cm}{\textit{#3}} \parbox{2cm}{#4} \parbox{2cm}{#5} \parbox{1.9cm}{#6}}

% Chapter 5
\newcommand{\redc}[1]{\textcolor{red}{#1}}
\newcommand{\bluec}[1]{\textcolor{blue}{#1}}
\newcommand{\ajout}[1]{\textcolor{blue}{#1}}
\newcommand{\ajoutplus}[1]{\textcolor{cyan}{#1}}

\newcommand{\hachure}[9]{
% Parametres :
% Coordonnees bas gauche (2 parametres) : (#1,#2)
% Coordonnees haut droit (2 parametres) : (#3,#4)
% Orientation : #5
%   1 : diagonale de pente 1
%  -1 : diagonale de pente -1
%   0 : horizontal
%   2 : vertical
% Nombre de pas horizontaux : #6
% Epaisseur du trait : #7
% Couleur : #8 (ex. green)
% Atténuation couleur : #9 (ex. 30)
\pgfmathsetmacro{\N}{#6-1}
\pgfmathsetmacro{\A}{#1}
\pgfmathsetmacro{\B}{#2}
\pgfmathsetmacro{\C}{#3}
\pgfmathsetmacro{\D}{#4}
\pgfmathsetmacro{\I}{(#3-#1)/#6}
\pgfmathsetmacro{\J}{(#4-#2)/#6}
\ifthenelse{\equal{#5}{1}}{
  \foreach \n in {0,...,\N}
    \foreach \m in {0,...,\N}
      {
        \pgfmathsetmacro{\X}{\A + ((0 + \n) * \I)}
        \pgfmathsetmacro{\Y}{\B + ((0 + \m) * \J)}
        \pgfmathsetmacro{\U}{\A + ((1 + \n) * \I)}
        \pgfmathsetmacro{\V}{\B + ((1 + \m) * \J)}
        \draw[#8!#9,#7] (\X, \Y)--(\U, \V);
      } 
  }{}
\ifthenelse{\equal{#5}{-1}}{
  \foreach \n in {0,...,\N}
    \foreach \m in {0,...,\N}
      {
        \pgfmathsetmacro{\X}{\A + ((1 + \n) * \I)}
        \pgfmathsetmacro{\Y}{\B + ((0 + \m) * \J)}
        \pgfmathsetmacro{\U}{\A + ((0 + \n) * \I)}
        \pgfmathsetmacro{\V}{\B + ((1 + \m) * \J)}
        \draw[#8!#9,#7] (\X, \Y)--(\U, \V);
      } 
  }{}
\ifthenelse{\equal{#5}{0}}{
  \foreach \n in {0,...,\N}
    \foreach \m in {0,...,\N}
      {
        \pgfmathsetmacro{\X}{\A + ((0 + \n) * \I)}
        \pgfmathsetmacro{\Y}{\B + ((0 + \m) * \J)}
        \pgfmathsetmacro{\U}{\A + ((1 + \n) * \I)}
        \pgfmathsetmacro{\V}{\B + ((0 + \m) * \J)}
        \draw[#8!#9,#7] (\X, \Y)--(\U, \V);
      } 
  }{}
\ifthenelse{\equal{#5}{2}}{
  \foreach \n in {0,...,\N}
    \foreach \m in {0,...,\N}
      {
        \pgfmathsetmacro{\X}{\A + ((0 + \n) * \I)}
        \pgfmathsetmacro{\Y}{\B + ((0 + \m) * \J)}
        \pgfmathsetmacro{\U}{\A + ((0 + \n) * \I)}
        \pgfmathsetmacro{\V}{\B + ((1 + \m) * \J)}
        \draw[#8!#9,#7] (\X, \Y)--(\U, \V);
      } 
  }{}
}

%Définition d'un pattern de type hachure
% \usetikzlibrary{patterns}
% \makeatletter
% \tikzset{hatch distance/.store in=\hatchdistance,hatch distance=5pt,hatch thickness/.store in=\hatchthickness,hatch thickness=5pt}

% \pgfdeclarepatternformonly[\hatchdistance,\hatchthickness]{north east hatch}% name
%     {\pgfqpoint{-\hatchthickness}{-\hatchthickness}}% below left
%     {\pgfqpoint{\hatchdistance+\hatchthickness}{\hatchdistance+\hatchthickness}}% above right
%     {\pgfpoint{\hatchdistance}{\hatchdistance}}%
%     {
%         \pgfsetcolor{\tikz@pattern@color}
%         \pgfsetlinewidth{\hatchthickness}
%         \pgfpathmoveto{\pgfqpoint{-\hatchthickness}{-\hatchthickness}}       
%         \pgfpathlineto{\pgfqpoint{\hatchdistance+\hatchthickness}{\hatchdistance+\hatchthickness}}
%         \pgfusepath{stroke}
%     }
% \pgfdeclarepatternformonly[\hatchdistance,\hatchthickness]{north west hatch}% name
%     {\pgfqpoint{-\hatchthickness}{-\hatchthickness}}% below left
%     {\pgfqpoint{\hatchdistance+\hatchthickness}{\hatchdistance+\hatchthickness}}% above right
%     {\pgfpoint{\hatchdistance}{\hatchdistance}}%
%     {
%         \pgfsetcolor{\tikz@pattern@color}
%         \pgfsetlinewidth{\hatchthickness}
%         \pgfpathmoveto{\pgfqpoint{\hatchdistance+\hatchthickness}{-\hatchthickness}}
%         \pgfpathlineto{\pgfqpoint{-\hatchthickness}{\hatchdistance+\hatchthickness}}
%         \pgfusepath{stroke}
%     }
% \makeatother
%~~~~~~~~~~~~~~~~~~~~~~~~~~~~~~~~~~~~~


% Chapter 7
\newcommand\pef[1]{(\ref{#1})}

\newcommand{\subscript}[1]{\textsubscript}

  %% hyphenation points for line breaks
%% Normally, automatic hyphenation in LaTeX is very good
%% If a word is mis-hyphenated, add it to this file
%%
%% add information to TeX file before \begin{document} with:
%% %% hyphenation points for line breaks
%% Normally, automatic hyphenation in LaTeX is very good
%% If a word is mis-hyphenated, add it to this file
%%
%% add information to TeX file before \begin{document} with:
%% %% hyphenation points for line breaks
%% Normally, automatic hyphenation in LaTeX is very good
%% If a word is mis-hyphenated, add it to this file
%%
%% add information to TeX file before \begin{document} with:
%% \include{localhyphenation}
\hyphenation{
anaph-o-ra
Dor-drecht
%FFI2016-76045-P-AEI/-MINEICO/-FEDE
}

\hyphenation{
anaph-o-ra
Dor-drecht
%FFI2016-76045-P-AEI/-MINEICO/-FEDE
}

\hyphenation{
anaph-o-ra
Dor-drecht
%FFI2016-76045-P-AEI/-MINEICO/-FEDE
}

    \bibliography{localbibliography}
    \togglepaper[23]
}{}


\begin{document}
\maketitle

\section{Introduction}

This chapter takes as a point of departure Bruening \& Tran’s (\citeyear{bruening2015a}) discussion of what it means to be a passive. As they point out, object promotion is not an essential feature of the passive. It is merely a side effect that may or may not occur. For example, the German and French impersonal passives in \ref{ex:authier:1a}--\ref{ex:authier:1b} do not involve object promotion and neither do the English expletive passives in \ref{ex:authier:1c}--\ref{ex:authier:1d}.


\begin{exe} % declares the example environment
    \ex\label{ex:authier:1} % sets the number, in this case (1)
    \begin{xlist} % declares a list of sub-examples
        \ex\label{ex:authier:1a} % sets the sub-example letter, in this case (a)
            \gll    Es wird gearbeitet. \\   % target language line
                    it becomes worked \\    % interlinear gloss
            \glt    `(People) are working.' % free translation
        \ex\label{ex:authier:1b}
            \gll    Il a \'{e}t\'{e} parl\'{e} de toi. \\
                    it has been talked about you \\
            \glt    `(Someone) talked about you.'
        \ex\label{ex:authier:1c} There were several studies conducted on teenage smoking. 
        \ex\label{ex:authier:1d} Near that site, there was believed to have been a blockhouse.
    \end{xlist}
\end{exe}


Further, there are cases involving object promotion, such as the \textit{tough}{}-movement construction in \ref{ex:authier:2b}, that have never been argued to instantiate passivization.


\begin{exe}
\ex\label{ex:authier:2}
\begin{xlist}
\ex\label{ex:authier:2a} It is difficult to find quality soccer players.
\ex\label{ex:authier:2b} Quality soccer players are difficult to find. 
\end{xlist}
\end{exe}


On the other hand, an essential property of passives is that the external argument of the predicate is either ‘missing’ or demoted to an oblique (cf. \citealt{perlmutter1983a}). An immediate consequence of this essential property is that verbs that do not take an external argument, such as unaccusatives, cannot, unlike unergatives, partake in any kind of passive. This is illustrated in \ref{ex:authier:3a} with Dutch impersonal passives and in \ref{ex:authier:3b} with English passives. 


\begin{exe}
\ex\label{ex:authier:3}
\begin{xlist}
\ex\label{ex:authier:3a}
\gll Er werd gelopen/*gevallen.\\
it became run/fallen\\ \jambox{(Dutch)}
\glt `(Someone/people) ran/fell.'
\ex \label{ex:authier:3b} The bridge was skied under by the contestants/*existed under by the trolls.
\end{xlist}
\citep[101]{perlmutter1984a}
\end{exe}

Thus, Bruening \& Tran argue, the Voice head that thematically relates the external argument to the event denoted by the VP comes in active and passive variants. The active variant of a Voice head projects the external argument in its specifier, while its passive variant does not, and instead existentially quantifies over it.

With this in mind, let us now consider those French \textit{on} constructions in which \textit{on} signals existential value for the understood Agent, as it does in \ref{ex:authier:4}. On this interpretation, \textit{on} has been referred to in the literature as arbitrary \textit{on} by \citet{egerland2003a}, ultra-indefinite \textit{on} by \citet{koenig1999a} and \citet{collins2017a}, and a-definite \textit{on} by \citet{koenig2000a}.


\begin{exe}
\ex\label{ex:authier:4}
\gll On a toussé ! \\
\textsc{on} has  coughed \\
\glt `Someone coughed/There was some coughing!'
\end{exe}

Existential \textit{on} constructions share with short passives a number of characteristics that are usually thought to be unique to implicit arguments. First, as illustrated in \ref{ex:authier:5}, existential \textit{on} constructions, just like the passive constructions in \ref{ex:authier:3}, are incompatible with unaccusative verbs, an observation that goes back to \citet{cinque1988a}.


\begin{exe} % declares the example environment
    \ex \label{ex:authier:5}
            \gll    On est tombé dans les escaliers. \\   % target language line
                    \textsc{on} is   fallen  in    the stairs \\    % interlinear gloss
            \glt    Can only mean: `We fell down the stairs.' % free translation
            \glt Cannot mean: `Someone fell down the stairs.'
\end{exe}


In other words, both existential \textit{on} and the implicit argument of a passive must correspond to an external argument. This property is also revealed by the fact that just like passive structures cannot be further passivized, existential \textit{on} constructions are incompatible with passivization, as \ref{ex:authier:6} shows.

\begin{exe} % declares the example environment
    \ex \label{ex:authier:6}
            \gll    On a    été    arrêté(s). \\   % target language line
                    \textsc{on} has been arrested \\    % interlinear gloss
            \glt    `*Someone was arrested/There were some arrests.' (OK `We were arrested.')' % free translation
\end{exe}



\ Second, unlike true indefinites like \textit{quelqu’un} ‘someone’ in \ref{ex:authier:7a}, the implicit argument of a short passive cannot bind pronouns (cf. \ref{ex:authier:7b}) and, as discussed in \citet[243]{koenig1999a}, neither can existential \textit{on} (cf. \ref{ex:authier:7c}).\footnote{ 
% FOOTNOTE 1 ==============================================================================================
The significance of this fact can be fully appreciated by observing that unlike existential \textit{on}, quasi-universal \textit{on} in generic sentences is able to bind pronouns.
\begin{exe}
\ex
\gll On$_i$ doit communiquer ces informations à son$_i$ banquier.\\
\textsc{on} must communicate this information to his banker\\
\glt `One must share this information with one’s bank.'
\end{exe}}
% END FOOTNOTE ============================================================================================


\begin{exe} % declares the example environment
    \ex \label{ex:authier:7}
    \begin{xlist} % declares a list of sub-examples
        \ex \label{ex:authier:7a}
            \gll    Quelqu’un$_{i}$ a     oublié     son$_{i}$ parapluie. \\   % target language line
                    someone    has  forgotten his  umbrella \\    % interlinear gloss
            \glt    `Someone forgot their umbrella.' % free translation
        \ex\label{ex:authier:7b}
            \gll    *Un os a été \textsc{imp}$_{i}$ donné à son$_{i}$ chien. \\    %%%%fix here
                    a bone has been {} given  to his dog\\
            \glt    Intended: `Someone gave their dog a bone.' (in passive form)\\
            (\textsc{imp} = implicit argument)
        \ex\label{ex:authier:7c}
             \gll    *On$_{i}$ a     oublié      son$_{i}$ parapluie. \\  
                    \textsc{on}  has forgotten  his  umbrella \\
            \glt    Intended: `Someone forgot their umbrella.'
    \end{xlist}
\end{exe}

Third, while true indefinites can generate discourse referents (cf. \ref{ex:authier:8a}), the implicit argument of a short passive cannot do so (cf. \ref{ex:authier:8b}) and, as \citet[241]{koenig1999a} observes, neither can existential \textit{on} (cf. \ref{ex:authier:8c}). 

\begin{exe} 
    \ex \label{ex:authier:8}
    \begin{xlist} 
        \ex \label{ex:authier:8a}
            \gll    Quelqu’un$_{i}$ a oublié un parapluie.  \\  
                    someone has forgotten an umbrella  \\  
            \glt {}
            \gll    Il$_{i}$ reviendra probablement le chercher.\\
                    he will-come-back probably it  to-fetch\\
            \glt    `Someone forgot an umbrella. They’ll probably come back to get it.'
        \ex\label{ex:authier:8b}
            \gll    Un parapluie a été \textsc{imp}$_{i}$ oublié.  \\
                    an  umbrella  has been {} forgotten \\
            \glt    {}
            \gll    *  Il$_{i}$ reviendra probablement le chercher.\\
                    {} he will-come-back probably        it  to-fetch \\
            \glt    Intended: `An umbrella was forgotten. Whoever forgot it will probably come back to get it.'
       \ex\label{ex:authier:8c}
            \gll    On$_{i}$ a    oublié     un parapluie. \\
                    \textsc{on} has forgotten an umbrella \\
            \glt    {}
            \gll    *  Il$_{i}$ reviendra probablement le chercher. \\
                    {} he will-come-back probably        it  to-fetch \\
            \glt    Intended: `Someone forgot an umbrella. Whoever forgot it will probably come back to get it.'
    \end{xlist}
\end{exe}


Fourth, unlike true indefinites, existential \textit{on} is scopally inert \citep{creissels2011a} and so is the implicit argument of a short passive \citep[148]{lasersohn1993a}; that is, both must take the narrowest scope with respect to any scope-bearing element present in the sentence in which they appear. For example, unlike the true indefinite \textit{quelqu’un }in \ref{ex:authier:9a}, existential \textit{on} in \ref{ex:authier:9b} and the implicit argument of the short passive in \ref{ex:authier:9c} must scope under\textbf{ }the adverbial \textit{deux fois cet été} ‘twice this summer’, which quantifies over the number of events denoted by the sentence. \ \ \ \ \ 


\begin{exe} 
\settowidth\jamwidth{(unambiguous)}
    \ex \label{ex:authier:9}
    \begin{xlist} % declares a list of sub-examples
        \ex \label{ex:authier:9a}
            \gll    Quelqu’un a    volé    mon vélo deux fois   cet  été.\\   % target language line
                    someone    has stolen my   bike two  times this summer\\ \jambox{(ambiguous)}
            \glt $\exists$ > 2 times i.e., There were two events of bike stealing initiated by the 						   same indeterminate individual. \\
                    2 times > $\exists$ i.e., There were two instances of some indeterminate 						              individual stealing my bike.% interlinear gloss
        \ex \label{ex:authier:9b}
        \gll On a    volé   mon vélo deux fois   cet   été.\\
         \textsc{on} has stolen my  bike two   times this summer \\\jambox{(unambiguous)}
        \glt 2 times > $\exists$ \textbf{only} i.e., There were two instances of some indeterminate individual stealing my bike.
       \ex \label{ex:authier:9c}
            \gll    Mon vélo a    été    volé   deux fois   cet  été. \\
            my   bike has been stolen two  times this summer \\\jambox{(unambiguous)}
            \glt 2 times > $\exists$ \textbf{only} i.e., There were two instances of some indeterminate individual stealing my bike. \\
            (Examples in \ref{ex:authier:9} adapted from \citealt[8]{creissels2011a})	
    \end{xlist}
\end{exe}

Consider next the fact, illustrated in \ref{ex:authier:10}, that both the implicit argument of a short passive and existential \textit{on} can serve as antecedents to a \textsc{pro} subject of a rationale clause.

\begin{exe} 
    \ex\label{ex:authier:10} 
    \begin{xlist}
        \ex\label{ex:authier:10a}
            \gll    Un coup de feu été \textsc{imp}$_{i}$ tiré pour \textsc{pro}$_{i}$ provoquer la   police. \\   % target language line
            a stroke of fire has-been {} fired to {} provoke the police \\    % interlinear gloss
            \glt    `A shot was fired to provoke the police.' % free translation
        \ex\label{ex:authier:10b}
            \gll    On$_{i}$ a tiré un coup de feu pour \textsc{pro}$_{i}$ provoquer la police. \\
                    \textsc{on} has fired a stroke of fire to provoke the police \\
            \glt    `Someone fired a shot to {} provoke the police.'
    \end{xlist}
\end{exe}


As shown in \citet{williams2017a}, there is indisputable evidence that the coreference relation in \ref{ex:authier:10a} cannot be one of argument control. For example, the same coreference relation can obtain across discourse, as shown in \ref{ex:authier:11a}. Further, as \ref{ex:authier:11b} illustrates, the \textsc{pro} of a rationale clause can denote an individual not named by any part of the target clause as long as it is viewed as responsible for the target fact. That is, in \ref{ex:authier:11b}, \textsc{pro} may be understood to refer to the organizers of the fundraiser.

\begin{exe} 
    \ex\label{ex:authier:11} 
    \begin{xlist}
        \ex\label{ex:authier:11a}
            \gll Un coup   de feu  a    été \textsc{imp}$_{i}$ tiré.  C’était probablement pour \textsc{pro}$_{i}$ provoquer la   police. \\   % target language line
            a    stroke of fire has been       {} fired it-was  probably to {} provoke the police\\ 
            \glt `A shot was fired. It was probably to provoke the police.'% free translation
        \ex\label{ex:authier:11b}
            \gll    {Il y a} eu  un {gala de bienfaisance} pour \textsc{pro} venir en aide à 6000 familles rurales pauvres. \\
                   there has-been a fundraiser to {} come in help to 6,000 families rural    poor\\
            \glt `There was a fundraiser to assist 6,000 poor rural families.'
    \end{xlist}
\end{exe}

And yet, as noted in \citet{jaeggli1986a} and \citet{landau2000a}, the implicit argument of a short passive cannot serve as the antecedent of the \textsc{pro} of a rationale clause if the latter is itself a passive. This constraint is illustrated in \ref{ex:authier:12}. The ungrammaticality of examples like \ref{ex:authier:12b} has been argued by \citet[310]{williams1985a}, \citet[9--16]{lasnik1988a} and \citet[318--321]{reed2014a} to follow from event control.\footnote{Event control refers to cases in which a \textsc{pro} inside a non-finite propositional adjunct refers to the event denoted by the main clause.} Simply put, an event can impress the board of directors \ref{ex:authier:12a} but events cannot be congratulated \ref{ex:authier:12b}. 

\begin{exe}
 \ex\label{ex:authier:12} 
\textit{The report was \textsc{imp}$_{i}$ carefully prepared...}
 \begin{xlist}
   \ex[] {\label{ex:authier:12a}
    \textit{\textsc{pro}$_{i}$ to impress the board of directors.}}
   \ex[*] {\label{ex:authier:12b}
   \textit{\textsc{pro}$_{i}$ to be congratulated by the board of directors.}}
    \end{xlist}
\end{exe}


What is of interest for our purposes, however, is that existential \textit{on} constructions behave exactly like short passives with respect to the constraint responsible for ruling out \ref{ex:authier:12b}. This is illustrated in \ref{ex:authier:13}.


\begin{exe}
\ex\label{ex:authier:13} 
 \gll On$_{i}$ a falsifié ces documents... \\
    \textsc{on} has falsified these documents\\
 \begin{xlist}
        \ex[] {\label{ex:authier:13a}
            \gll pour \textsc{pro}$_{i}$ obtenir accès à nos fichiers.\\ 
                 to {} gain access to our files \\}
        \ex[*] {\label{ex:authier:13b}
            \gll  pour \textsc{pro}$_{i}$ être autorisé à consulter nos fichiers.\\
                  to {} be allowed to consult our files\\}
\end{xlist}
\end{exe}


There is, however, in the midst of these similarities, one notable difference between existential \textit{on} and the implicit argument found in passives, which is that only the former must be understood as being [+ human]. For example, \ref{ex:authier:14} can only be understood to mean that someone, not something, killed Patrick whereas the short passive in \ref{ex:authier:15} does not so restrict the interpretation of the implicit external argument, which can be understood as being inanimate (e.g., a stray bullet).

\begin{exe}
\ex\label{ex:authier:14} 
\gll On a tué Patrick.\\
\textsc{on} has killed Patrick\\ 
\glt `Someone/*something killed Patrick.'
\end{exe}

\begin{exe}
\ex\label{ex:authier:15} 
\gll Patrick a été tué.\\
Patrick has been killed\\
\glt `Patrick was killed (by someone/something).'
\end{exe}

This difference should not, however, be taken to indicate that the external argument in existential \textit{on} constructions is fundamentally different from that found in passives for at least two reasons. First, as \citet[43]{gaatone1994a} points out, the [+ human] restriction on the understood external argument is also present in those French impersonal passives that do not alternate with passives involving promotion. Gaatone illustrates this constraint with the contrast in \ref{ex:authier:16}.

\begin{exe}
\ex\label{ex:authier:16} 
\begin{xlist} % declares a list of sub-examples
        \ex[]{\label{ex:authier:16a}
\gll  Il a déjà été répondu à ces questions (par le ministre).\\
it has already been answered to these questions (by the minister)\\
\glt `These questions have already been answered (by the minister).'\\}

        \ex[*]{ \label{ex:authier:16b}
        \gll Il a été répondu à notre attente (par les résultats obtenus).\\
        it has been met to our expectation (by the results obtained)\\
       \glt  `Our expectations were met (by the obtained results).'\\}
\end{xlist}
\end{exe}


Second, passives involving promotion in Hebrew display the same [+ human] requirement on the interpretation of their implicit argument, as noted by \citet{meltzer-asscher2012a}, who says of examples like \ref{ex:authier:17} that speakers “agree that it is only possible that someone sank the boat, not something.”

\begin{exe}
\ex\label{ex:authier:17} 
\gll ha-sira hutbe’a \\
the-boat was-sunk\\\jambox{\citep{meltzer-asscher2012a}}
\glt `The boat was sunk.'\\
\end{exe}

\section{A new hypothesis}

As the preceding section has shown, the implicit external argument of a short passive and the external argument of an existential \textit{on} construction share a substantial number of properties, summed up in \ref{ex:authier:18}, and this must be explained.


\begin{exe}
\ex\label{ex:authier:18} 
\begin{xlist}
    \ex\label{ex:authier:18a}  Incompatibility with unaccusatives and passives;
    \ex\label{ex:authier:18b}  Inability to antecede pronouns;
    \ex\label{ex:authier:18c}  Scopal inertness (lowest scope only);
    \ex\label{ex:authier:18d}  Inability to control the \textsc{pro} of a passive rationale clause.
\end{xlist}
\end{exe}
\todo[]{Convert to list instead of example?}

One approach to existential \textit{on} that seeks to account for its discourse inertness has been to assume that the sentences in which it appears are in the active voice but that existential \textit{on} is an “ultra-indefinite” \citep{koenig1999a} or “a-definite” \citep{koenig2000a} clitic pronoun. That is, in the two papers just mentioned, it is argued that existential \textit{on} is a pronoun that is neither definite, nor indefinite; it is ultra-indefinite. To explain, while definites function as discourse markers anchored to a previously introduced entity and indefinites introduce new discourse markers, ultra-indefinites are taken to satisfy a predicate’s argument position without generating a discourse marker into the Discourse Representation Structure of a sentence, hence they are discursively inert. While this proposal accounts for the discourse inertness of existential \textit{on}, it is not immediately obvious whether it can predict its scope inertness property \ref{ex:authier:18b} relative to indefinites like \textit{quelqu’un}.
\footnote{The same problem arises in conjunction with Collins’s (\citeyear{collins2017a}) reinterpretation of ultra-indefinites as syntactically projected pronouns with no phi-feature. On this account, the discourse inertness of existential \textit{on} is accounted for by the Pronominal Agreement Condition of \citet[92]{collins2012a} given in \ref{ex:authier:i}.
\begin{exe}
\ex\label{ex:authier:i} If P is a non-expletive pronominal, then P agrees with some source in those phi-features for which P is not inherently valued.
\end{exe}
\par Thus, \ref{ex:authier:i} predicts that phi-feature endowed pronouns cannot be anteceded by the implicit argument of a short passive or by existential \textit{on} because by assumption, these are ultra-indefinites with no phi-features. But again, it remains unclear how the scope inertness of ultra-indefinites might follow. }

In this paper, we aim to take a different approach to this issue. Specifically, we will explore the possibility that existential \textit{on} constructions are not what they appear to be; that is, they are not active sentences with an ultra-indefinite subject pronoun. We will suggest instead that the properties of existential \textit{on} and those of the implicit argument of the passive are in some way linked to the kind of Voice head present in the sentences in which they occur. More specifically, we will argue that existential \textit{on} is dependent on the presence of a non-active Voice head. An immediate advantage of this approach is that, unlike theories that rely on particular lexical semantic properties inherent to the ultra-indefinite nature of existential \textit{on}, it can account for the fact that existential \textit{on} is incompatible with unaccusatives and passives \ref{ex:authier:18a}. The ultra-indefinite pronoun analysis has to stipulate that ultra-indefinites are restricted to underlying thematic subjects, a fact that remains mysterious given that all other thematic pronouns, including \textsc{pro} and pro can be derived subjects. On the assumption that existential \textit{on} constructions are in a non-active voice, on the other hand, they are predicted, along with passives, to be incompatible with unaccusatives because they crucially involve demotion of an external argument. Further, the fact that existential \textit{on} is incompatible with passivization \ref{ex:authier:6} can be attributed to the impossibility for a sentence to express two distinct voices (i.e., the non-active voice signaled by existential \textit{on} and the passive voice encoded by the \textit{be} auxiliary and the passive participle). In other words, the ungrammaticality of \ref{ex:authier:6} is on a par with that of \ref{ex:authier:19c}, which combines the middle voice in \ref{ex:authier:19a} with the passive voice in \ref{ex:authier:19b}.

\begin{exe}
\ex\label{ex:authier:19} 
\begin{xlist} % declares a list of sub-examples
        \ex\label{ex:authier:19a} 
        \gll    Ce vin se boit frais. \\
                this wine SE drinks chilled\\\jambox{(middle)}

        \ex\label{ex:authier:19b} 
        \gll    Ce vin est bu frais. \\
                this wine is drunk chilled\\\jambox{(passive)}
        \ex\label{ex:authier:19c} 
        \gll    *Ce vin s’est bu frais. \\
                this wine SE-is drunk chilled\\\jambox{(middle + passive)}
\end{xlist}
\end{exe}


But is there independent evidence that suggests that existential \textit{on} sentences are not active sentences? \citep{merchant2013a} argues, based on data concerning the availability of voice mismatches in ellipsis, that ellipsis requires identity between syntactic phrase markers. The data in question show that while voice mismatches are licit in low ellipsis such as VP-ellipsis in \ref{ex:authier:20}, they are prohibited with high ellipses such as sluicing, as shown in \ref{ex:authier:21}, and stripping, as shown in \ref{ex:authier:22}.


\begin{exe}
\ex\label{ex:authier:20} 
\begin{xlist} % declares a list of sub-examples
        \ex\label{ex:authier:20a}  \textit{This should have been brought up long ago
                but apparently,\\ nobody did Ø.}\\
                Ø = <bring this up long ago>
        \ex\label{ex:authier:20b}   \textit{Nobody here can cook venison the way it should be Ø.}\\
                Ø = <cooked>
\end{xlist}
\end{exe}

\begin{exe}
\ex\label{ex:authier:21} 
\begin{xlist} % declares a list of sub-examples
        \ex[*]{\label{ex:authier:21a} \textit{They were robbed, but they don’t know who Ø.}\\
               Ø = <robbed them>}
        \ex[*] {\label{ex:authier:21b} \textit{Someone robbed them, but they don’t know who by Ø.}\\
               Ø = <they were robbed>}
\end{xlist}
\end{exe}

\begin{exe}
\ex\label{ex:authier:22} 
\begin{xlist} % declares a list of sub-examples
        \ex[] {\label{ex:authier:22a} \textit{Someone bought roses for the wedding, but not (*by) Joe.}}
        \ex[] {\label{ex:authier:22b} \textit{Roses were bought for the wedding, but not *(by) Joe.}}
\end{xlist}
\end{exe}

This, he argues, follows from the fact that ellipsis is subject to identity between phrase markers. That is, in VP-ellipsis, since VP is structurally lower than VoiceP, the Voice head is not included in the target of ellipsis and is therefore not subject to elliptical identity. In bigger ellipses that target larger nodes that contain VoiceP, such as sluicing and stripping, Voice is part of the elided structure and is therefore subject to elliptical identity. Thus, the geometry of licit and illicit voice mismatches is as in \ref{ex:authier:23}.

% Diagram 1
%%% Original:
% \begin{exe}
\ex
\jtree[scaleby=1.5]
\! = {} !a !b
    <right>[branch=\etcbranch]{XP}
    <left>{} ^<right>[branch=\etcbranch]{VoiceP}
    <left>{Voice} ^<right>[branch=\etcbranch]{YP}.
\!a = <right>[branch=\blank,xunit=4]{$\rightarrow ~\varnothing$: voice mismatch \textit{disallowed}}.
\!b = <right>[branch=\blank,xunit=5.75,yunit=5]{$\rightarrow ~\varnothing$: voice mismatch \textit{allowed}}.
\endjtree
\jambox{\citep[89]{merchant2013a}}
\end{exe}
%\vspace{0.5cm}
\begin{exe} 
\ex\label{ex:authier:23}
\begin{forest}  
where level=1{s sep=2.5em}{},
[ , calign=first
  [,phantom]
  [XP, edge=dotted
    []
    [VoiceP, edge=dotted
      [Voice]
      [YP, edge=dotted]{\draw[draw=none] (.west)--(3,-3.1) node[anchor=west,align=right]{$\rightarrow \varnothing$: voice mismatch \textit{allowed}};}
    ]
   ]{\draw[draw=none] (.west)--(1.5,-0.95) node[anchor=west,align=right]{$\rightarrow \varnothing$: voice mismatch \textit{disallowed}};}
]
\end{forest}\\
\citep[89]{merchant2013a}
\end{exe}

Merchant’s theory of ellipsis gives us a way to test whether existential \textit{on} sentences are in a voice different from the active voice. This can be done by using an existential \textit{on} sentence as an antecedent for a high ellipsis while forcing the elided material to be in the active voice. If this results in an illicit voice mismatch, we will have evidence that existential \textit{on} sentences are not active sentences. The ungrammatical status of the examples involving sluicing in \ref{ex:authier:24b} and stripping in \ref{ex:authier:25b} suggests that this is indeed the case. These are to be contrasted with their grammatical active counterparts with an existentially quantified subject in \ref{ex:authier:24a} and \ref{ex:authier:25a}. 



\begin{exe}
\ex\label{ex:authier:24} 
\begin{xlist} % declares a list of sub-examples
        \ex[] {\label{ex:authier:24a} 
        \gll Je sais pas qui Ø, mais quelqu’un a oublié d’éteindre la lumière hier soir.\\
        I know not who {} but someone has forgotten      of-to-turn-out the light yesterday evening\\
    \glt Ø = <a oublié d’éteindre la lumière hier soir>\\}
        \ex[*?] {\label{ex:authier:24b} 
        \gll Je sais pas qui Ø, mais on a oublié d’éteindre la lumière hier soir.\\
        I know not who {} but \textsc{on} has forgotten        of-to-turn-out the light yesterday evening\\
    \glt Ø = <a oublié d’éteindre la lumière hier soir>\\
    \glt `I don’t know who, but someone forgot to turn off the light last night.'\\}
\end{xlist}
\end{exe}

\begin{exe}
\ex\label{ex:authier:25} 
\begin{xlist} % declares a list of sub-examples
        \ex[] {\label{ex:authier:25a}
        \gll  Quelqu’un a crié au secours, mais pas Claire.\\
       someone has shouted to-the rescue but not Claire\\}
        \ex[*] {\label{ex:authier:25b}
        \gll On a crié au secours, mais pas Claire.\\
       \textsc{on} has shouted to-the rescue but not Claire\\
        \glt `Someone shouted for help, but not Claire.'\\}
\end{xlist}
\end{exe}

Importantly, proponents of the ultra-indefinite theory cannot argue that ultra-indefinite pronouns do not support sluicing or stripping for semantic reasons since the implicit argument of a short passive allows both sluicing and stripping provided that the voice of the elided material is also in the passive voice \ref{ex:authier:26} nor can they argue that some factor other than elliptical voice matching is at work in \ref{ex:authier:24b}, given that removing the ellipsis by using \textit{do}{}-\textit{it} VP-anaphora restores grammaticality \ref{ex:authier:27}.


\begin{exe}
\ex\label{ex:authier:26} 
\begin{xlist} % declares a list of sub-examples
        \ex \label{ex:authier:26a}
        \gll  Ce document a été modifié, mais je sais pas par qui.\\
      this document has been modified but I know not by who\\
        \ex \label{ex:authier:26b}
        \gll Ce document a bien été modifié, mais pas par moi!\\
      this document has indeed been modified but not by me\\
\end{xlist}
\end{exe}

\begin{exe}
\ex\label{ex:authier:27} 
        \gll   Je sais pas qui a fait ça, mais on a renversé les poubelles devant ta porte.\\
      I know not who has done that but \textsc{on} has knocked-down the {trash cans} in-front-of your door\\
\end{exe}

Thus, we now have evidence that existential \textit{on} constructions are not in the active voice. In the next section, we offer a theoretical analysis of the relation between non-active voice and existential \textit{on} constructions.


\section{Analysis}

In this section, we will argue that existential \textit{on} is not a pronoun but rather, the morphological reflex of T failing to Agree (in the sense of \citealt{preminger2009a}, \citeyear{preminger2014a}) due to the presence of a non-active Voice head that selects an unsaturated \textit{v} projection.\footnote{Unlike passives in French, existential \textit{on} constructions cannot express the external argument as a \textit{by}{}-phrase. This, however, has no impact on our claim that they instantiate a non-active voice, due to the fact that the availability of a \textit{by}{}-phrase cannot be considered to be an essential characteristic of non-active voices in general. For example, in many languages, such as Latvian and Classical Arabic, passives do not allow \textit{by}{}-phrases (see \citet{comrie1977a}; \citet[602]{jaeggli1986a}; \citet[]{siewierska1984a}). Furthermore, the middle voice also disallows \textit{by}{}-phrases (see \citealt{bruening2013a} and references cited there).} That existential \textit{on} is not a pronoun is not as strange as it may seem. There are, in French, other pronoun-like elements that, on closer inspection, turn out to not be pronouns at all. For example, the question marker \textit{tu} in Québec French, illustrated in \ref{ex:authier:28}, is homophonous with the pronoun \textit{tu} ‘you’ and the middle voice marker \textit{se} in \ref{ex:authier:29a} shares the same phonological form as the reflexive pronoun \textit{se} ‘herself’ in \ref{ex:authier:29b}. 

\begin{exe}
\ex\label{ex:authier:28} 
\begin{xlist} % declares a list of sub-examples
        \ex[\%] {\label{ex:authier:28a}
        \gll   Tu as-tu grossi? \\
                 you has-Q gotten-bigger\\
        \glt    `Did you put on weight?'\\
        (Q = question particle)}
        \ex[\%] {\label{ex:authier:28b}
        \gll  Combien que ça coûte-tu?\\
                how-much that it costs-Q\\
        \glt    `How much does it cost?'}
\end{xlist}
\end{exe}

\begin{exe}
\ex\label{ex:authier:29} 
\begin{xlist}
        \ex\label{ex:authier:29a} 
        \gll    La Seine se voit bien d’ici.\\
                the Seine SE sees well from-here\\
        \glt    `The Seine is easily seen from here.'\\
        \ex\label{ex:authier:29b} 
        \gll    Véro s’est coupée.\\
                Véro SE-is cut\\
        \glt    `Véro cut herself.'\\
\end{xlist}
\end{exe}


Regarding the middle voice marker \textit{se} in \ref{ex:authier:29a}, it is important to point out that it cannot be a pronoun even though it behaves like a typical French clitic in that it attaches to the highest verbal element in the clause in which it appears. The reason why it cannot be a pronoun is that it does not correspond to the thematic object and, because the thematic object has undergone promotion, it cannot possibly be seen as a derived subject either. Thus, middle \textit{se} is a voice affix with clitic-like properties. Our claim is therefore that existential \textit{on} is an affix in the same sense. This, of course, immediately raises the question of how the EPP is satisfied in an existential \textit{on} construction, a question to which we will return shortly.

Before we do so, however, we would like to introduce an additional set of data that involve what appears to be the silent equivalent of existential\textit{ on} in ECM contexts. ECM contexts are, in French, restricted to the infinitival complements to perception verbs and \textit{laisser} ‘let’.\footnote{As well as \textit{faire} ‘make’ in Canadian French (cf. \citealt{reed1992}). \todo[inline]{Bibliography entry missing}} The sentence in \ref{ex:authier:30} illustrates the phenomenon.

\begin{exe}
\ex\label{ex:authier:30} 
        \gll J’ai entendu Géraldine tousser.\\
             I-have heard Géraldine cough\\
        \glt `I heard Géraldine cough.'\\
\end{exe}


Interestingly, ECM infinitivals can appear without a lexical subject and, in such cases, the external argument of the infinitive must be understood as existentially quantified, as \ref{ex:authier:31} makes clear.

\begin{exe}
\ex\label{ex:authier:31} 
\begin{xlist}
        \ex\label{ex:authier:31a} 
        \gll   J’ai entendu [ø tousser].\\
               I-have heard {} to-cough\\
        \glt    `I heard someone cough.'\\
        \ex\label{ex:authier:31b} 
        \gll La seule fois que j’ai vu [ø pendre un homme], c’était en Irak.\\
        the only time that I-have seen {} to-hang a man        it-was in Iraq\\
        \glt    `The only time I saw someone hang a man/a man being hanged was in Iraq.'\\
\end{xlist}
\end{exe}

The question then arises as to what, if anything, occupies the subject position of the infinitives in \ref{ex:authier:31}. It cannot be pro because French is not a null subject language. It cannot be \textsc{pro} either because under any theory of control, this would result in coreference between \textsc{pro} and the matrix subject (i.e., \ref{ex:authier:31a} would mean \textit{I heard myself coughing}, which is not a possible interpretation).\footnote{Alternatively, if \ref{ex:authier:31} were assumed to instantiate NOC, \textsc{pro} would have either a quasi-universal generic reading or would have to refer (logophorically) to some specific person whose thoughts or speech is being reported (see \citealt{williams1992a} and \citealt{reed2018a} for discussion). However, neither of these two readings turns out to be available. } And yet such infinitives display all of those restrictions associated with \textit{on} in tensed clauses listed in \ref{ex:authier:18}. That is, the construction illustrated in \ref{ex:authier:1} is impossible if the infinitive verb is an unaccusative \ref{ex:authier:32a} or a passive \ref{ex:authier:32b}, and, as \ref{ex:authier:32c} illustrates, the embedded external argument must be [+ human]. (The understood agent of destruction in \ref{ex:authier:25c} could not be a fire, for example.)\todo{There is no example (25c)}

\begin{exe}
\ex\label{ex:authier:32} 
\begin{xlist}
        \ex[*]{\label{ex:authier:32a} 
        \gll   J’ai entendu [ø tomber dans les escaliers].\\
               I-have heard {} to-fall in the stairs\\
        \glt    Intended: `I heard someone fall down the stairs.'\\}
        \ex[*]{\label{ex:authier:32b} 
        \gll J’ai vu [ø être arrêté hier].\\
        I-have seen {} to-be arrested yesterday\\
        \glt  Intended: `I saw someone get arrested yesterday.'\\}
        \ex[]{\label{ex:authier:32c} 
        \gll Ils ont regardé détruire tout ce qu’ils possédaient sans
 réagir.\\
        they have watched to-detroy all that that-they owned
 without to-react\\
        \glt  `They watched everything they owned being destroyed without reacting.'\\}
\end{xlist}
\end{exe}

Further, the understood embedded subject cannot antecede a pronoun \ref{ex:authier:33a} and it displays scope inertness \ref{ex:authier:33b}.

\begin{exe}
\ex\label{ex:authier:33} 
\begin{xlist}
        \ex[*]{\label{ex:authier:33a} 
        \gll J’ai vu [ø vandaliser sa propre voiture].\\ \\
        \glt Intended: `I saw someone vandalize their own car.'\\}
        \ex[]{\label{ex:authier:33b} 
        \gll J’ai vu [ø vandaliser cette statue deux fois cet été].\\
         I-have seen {} to-vandalize this statue two times this summer\\
        \glt 2 times > $\exists$ \textbf{only} i.e., I saw two instances of some indeterminate individual vandalizing this statue.\\}
\end{xlist}
\end{exe}



Thus, it seems reasonable to characterize the infinitives in \ref{ex:authier:31} and \ref{ex:authier:33b} as instantiating the “silent version” of the tensed existential \textit{on} construction. The next step is, of course, to devise an account that explains why existential \textit{on }is either silent or absent in infinitival clauses. If we analyze \textit{on} as a subject pronoun in tensed clauses, then we must conclude that it must surface as a silent pronominal in infinitivals. However, as we previously pointed out, this silent pronoun can neither be \textsc{pro} nor pro, which leads us to an impasse. Could \textit{on} then be the lexicalization of a non-active Voice head endowed with clitic-like properties that require that it appear on whatever element lexicalizes T in tensed clauses? While not implausible, this hypothesis leads us to stipulate that this type of Voice head must be overt in tensed clauses and covert in infinitivals, which is unexpected given that in passives and middles, voice morphology is realized in the same way in tensed and untensed clauses.\footnote{We are indebted to a reviewer for pointing this out.}

There is, however, another way of looking at these facts. An obvious difference between tensed T and untensed T is that only the former is endowed with unvalued \ ${\varphi}${}-features and acts as a probe with respect to Agree. Let us assume, drawing in part from the theory of passives advocated by \citet{bruening2013a}, that existential \textit{on} constructions contain a non-active Voice head that selects a projection of agentive \textit{v} that has not yet projected its external argument and that therefore the complement to Voice is an unsaturated \textit{v} projection. Let us further assume that this non-active Voice head has the property of requiring all of the arguments in its complement to be saturated and that it saturates the unprojected external argument of \textit{v} by existentially binding it. Assuming finally that unprojected arguments have no valued ${\varphi}${}-features, when tensed T is introduced, it will act as a probe with respect to ${\varphi}${}-agree, given that it has unvalued ${\varphi}${}-features, but finds no target/goal since unprojected arguments are not eligible goals for the purposes of ${\varphi}${}-agreement. Under \citet{chomsky2001a} conception of Agree, this should lead to a crash. This is because, in Chomsky’s proposal, all and only uninterpretable features are unvalued and in accordance with Full Interpretation, unvalued features, which semantics cannot interpret, must be eliminated in the narrow syntax. This is done though their deletion, a prerequisite for which is valuation via Agree. This view of Agree is, however, challenged by Preminger (\citeyear{preminger2009a}, \citeyear{preminger2014a}). On the basis of empirical data from Hebrew \citep{preminger2009a} and data from Basque, Kichean and Zulu \citep{preminger2014a}, he argues that the correct characterization of the relation between ${\varphi}${}-agreement and (un)grammaticality is as in \ref{ex:authier:34}.

\begin{exe}
\ex\label{ex:authier:34} 
 ``You can fail, but you must try'' \citep[32]{preminger2009a}\\
 Applying $\phi$-agreement to a given structure is obligatory; but if the structure happens to be such that $\phi$-agreement cannot culminate successfully, this is an acceptable outcome.
\end{exe}


Preminger further argues that when ${\varphi}${}-agreement fails, the unvalued features on the probe remain unvalued, retaining their preexisting or default values.\footnote{It should be mentioned that Preminger also discusses some cases in which failure to agree leads to ungrammaticality. In those cases, however, the Agree relation stands in a feeding relation with a movement operation. This suggests that successful Agree is a prerequisite for (though not necessarily the trigger of) movement.} Thus, default agreement is the morphology that surfaces when a probe fails to find a goal bearing the appropriate valued features \citep[137]{preminger2014a}. Adopting Preminger’s assumptions, we would like to suggest that so-called existential \textit{on} is, in fact, the morphological realization of tensed T failing to ${\varphi}${}-agree with a goal, resulting in a default 3\textsuperscript{rd} person singular affix on T. This immediately explains why ECM infinitivals do not feature \textit{on}, namely because untensed T is inert with respect to ${\varphi}${}-agreement. Under our proposal then, the derivation for a (tensed) existential \textit{on} construction is therefore assumed to be as in \ref{ex:authier:35}.

% Diagram 2
%%% Original:
% \begin{exe}
	\ex On a vandalis\'{e} le Louvre.
	
	\jtree[scaleby=1.5, labelgap=0]
	\! = {TP}
	<left>{T$^{0}$}({u$\varphi$ = \textit{on + a}})             ^<right>{VoiceP}
	<left>{Voice$^{0}$}({$\exists x$})         ^<right>{\textit{v}P}
	<left>{$v$}({$\theta$ Agent($x$)})             ^<right>{VP}
	<left>{V}({vandalis\'{e}})                   ^<right>{DP}
	<left>{D}({le})                   ^<right>{NP}({Louvre}).
	\endjtree
\end{exe}
%\vspace{0.5cm}
\begin{exe}
\ex\label{ex:authier:35}
On a vandalis\'{e} le Louvre.\\
\begin{forest}  
[TP
  [T$^0$\\{u$\varphi$ = \textit{on + a}}, align=center, base=bottom]
  [VoiceP
    [Voice$^0$\\$\exists x$, align=center, base=bottom]
    [\textit{v}P
      [\textit{v}\\$\theta$ Agent(\textit{x}), align=center, base=bottom]
      [DP
        [D\\le, align=center, base=bottom]
        [NP\\Louvre, align=center, base=bottom]]
    ]
  ]
]
\end{forest}
\end{exe}


In \ref{ex:authier:35}, V merges with the object DP \textit{le Louvres}, which is endowed with lexically valued ${\varphi}${}-features and an unvalued Case feature. The VP thus projected in turn merges with an agentive \textit{v} endowed with unvalued ${\varphi}${}-features and an unprojected external argument (symbolized as ${\theta}$Agent (x)). Acting as a ${\varphi}${}-probe, \textit{v} agrees with the object DP, resulting in the valuation of its unvalued ${\varphi}${}-features and in the valuation of the unvalued Case feature of the object DP. Next, a non-active Voice head is merged that existentially binds the unprojected external argument introduced by v. Finally, VoiceP is merged with tensed T, which, being endowed with unvalued ${\varphi}${}-features, attempts to undergo Agree with a goal bearing valued ${\varphi}${}-features. Since all of the features of the object DP have been valued, making it inaccessible to further computation, and since the unprojected external argument has no ${\varphi}${}-features, the Agree relation fails and T retains its preexisting or default ${\varphi}${}-values, which end up being lexicalized in the phonological component as \textit{on}. 
One final issue remains to be addressed, namely how existential \textit{on} sentences satisfy the EPP. We adopt the view defended by \citet{holmberg2000a}, \citet{merchant2001a} and \citet{landau2007a} that EPP is not a narrow syntactic condition, but rather, a PF condition requiring the presence of an overt element under particular circumstances. More specifically, we follow \citet{mcfadden2018a} in assuming that EPP is a PF-constraint and that the complement of a phase defining head, which is a spellout domain, constitutes an intonational phrase which must be phonologically overt according to An’s (2007) Intonational Phrase Edge Generalization (IPEG) stated in \ref{ex:authier:36}.


\begin{exe}
\ex\label{ex:authier:36} 
 \textit{Intonational Phrase Edge Generalization} \textit{(IPEG)}\\
 The edge of an IntP cannot be empty (where the notion of edge encompasses the specifier and the head of the relevant syntactic constituent).\\
\citep[61]{an2007a}
\end{exe}


Assuming with \citet{haegeman2017a} that FinP is part of the C-domain, the intonational phrase of a tensed on existential construction corresponds to TP, a situation that we illustrate in \ref{ex:authier:37}.

% Diagram 3
%%% Original:
% \begin{exe}
	\ex 
	\jtree[scaleby=1.5, labelgap=0]
	\! = {\textit{C-field}} !a
	<left>{Fin$^{0}$} ^<right>{TP}
	<left>{T${0}$}({\textit{on}}) ^<right>{VoiceP}
	<left>{Voice$^{0}$} ^<right>{\textit{v}P}.
	\!a = <right>[branch=\blank,xunit=5]{= \textbf{ Spellout domain/Intonational phrase}}.
	\endjtree
\end{exe}
%\vspace{0.5cm}
\begin{exe} 
\ex\label{ex:authier:37}
\begin{forest}  
[\textbf{C-field}
  [Fin$^0$]
  [TP
    [T0\\\textbf{on}, align=center, base=bottom]
    [VoiceP
      [Voice$^0$]
      [\textit{v}P]
   ]{\draw[draw=none] (.west)--(1.5,-0.96) node[anchor=west,align=right]{{\textbf{= Spellout domain/Intonational phrase}}};}
]]
\end{forest}
\end{exe}

We propose that the \textit{on} phonological realization of the default ${\varphi}${}-values of T resulting from a failure to agree satisfies the IPEG in that the IPEG has nothing to do with subjects or the [Spec, TP] position but is a PF requirement that the left edge of TP be overt because in \ref{ex:authier:37}, it is aligned with the intonational phrase. It therefore follows that overt \textit{on} in T satisfies the IPEG. As for ECM infinitivals containing a non-active Voice head, we assume that they are TPs lacking a CP layer, from which it follows that they do not constitute a spellout domain and are therefore not parsed as intonational phrases. The IPEG is therefore not relevant in this case. This allows the non-agreeing infinitival T head to be phonologically null. \ 


\section{Concluding remarks}

To sum up, our analysis of existential \textit{on} sentences captures three syntactic properties of the construction that remain unexplained under the ultra-indefinite approach. First, as the reader will recall, existential \textit{on} is incompatible with unaccusatives. This property follows from the fact that the non-active Voice head present in existential \textit{on} constructions selects a projection of \textit{v} that has not yet projected its external argument. If we assume a uniform structural architecture for transitive and unaccusative clauses, unaccusative \textit{v}Ps are headed by a \textit{v} that does not have the ability to project an external argument, hence it cannot be selected by the type of non-active Voice head found in the existential \textit{on} construction. Second, the fact that passive sentences with \textit{on} can never be interpreted as having an existentially quantified external argument follows from the fact there can only be one Voice head per clause. Third, assuming with \citet{merchant2013a} that ellipsis requires identity between phrase markers, we explain why existential \textit{on} sentences, being in a non-active voice, cannot serve as the antecedent for a high ellipsis such as sluicing or stripping when the elided material is in the active voice.

This leaves us with two properties of existential \textit{on} constructions with respect to which the ultra-indefinite account and the non-active voice account appear to roughly fare the same. Regarding discourse inertness, we argued that unprojected arguments are not specified for ${\varphi}${}-features. A consequence of this is that they cannot agree in ${\varphi}${}-features with ${\varphi}${}-specified pronouns in the discourse. A somewhat similar argument is given in \citet{collins2017a} who defines ultra-indefinites as pronouns with no ${\varphi}${}-features (cf. ft.2). However, Collins assumes that implicit arguments are \textit{projected} pronominal arguments that are lexicalized as \textit{on} (or as pro in short passives), a view which we have argued against here. Finally, the scope inertness of existential \textit{on} remains unaccounted for under any theory, as far as we can tell. We could perhaps derive this property from the hypothesis that unprojected quantificational arguments cannot undergo QR, although this move would no doubt have ramifications in other areas of the grammar. This is an issue that we will relegate to further work.


\section*{Acknowledgements}
  We gratefully acknowledge the insightful comments and suggestions made by members of the LSRL 49 audience at the University of Georgia and by two anonymous referees. All remaining errors are our sole responsibility.
%\citet{Nordhoff2018} is useful for compiling bibliographies

\printbibliography[heading=subbibliography,notkeyword=this]

\end{document}
