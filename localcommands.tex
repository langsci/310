\newcommand*{\orcid}{}

\makeatletter
\let\theauthor\@author
\makeatother

\papernote{\scriptsize\normalfont
    \theauthor.
    \titleTemp. 
    To appear in: 
    Chad Howe and Pilar Chamorro and Timothy Gupton and Margaret Renwick.
    Theory, Data, and Practice: Selected papers from the 49th Linguistic Symposium on Romance Language
    Berlin: Language Science Press. [preliminary page numbering]
}

% Workaround for subscripts with capital letters
\newcommand{\capsub}[1]{\ensuremath{_\text{#1}}}

% Chapter 10: Table-like presentation within example environment
% classical latin > {*}late latin > old french  earlier > later   gloss
\newcommand{\montanoboxi}[7]{\parbox{2cm}{#1} > {#2}\parbox{2cm}{#3} > \parbox{1.5cm}{\textit{#4}} \parbox{1.2cm}{#5}\ > \parbox{1.2cm}{#6} \parbox{1.5cm}{#7}}
% {*}latin > earlier OF [ipa] > early OF   gloss
\newcommand{\montanoboxii}[6]{{#1}\parbox{1.9cm}{\textit{#2}} > \parbox{1.3cm}{\textit{#3}} \parbox{2cm}{#4} \parbox{2cm}{#5} \parbox{1.9cm}{#6}}

% Chapter 5
\newcommand{\redc}[1]{\textcolor{red}{#1}}
\newcommand{\bluec}[1]{\textcolor{blue}{#1}}
\newcommand{\ajout}[1]{\textcolor{blue}{#1}}
\newcommand{\ajoutplus}[1]{\textcolor{cyan}{#1}}

\newcommand{\hachure}[9]{
% Parametres :
% Coordonnees bas gauche (2 parametres) : (#1,#2)
% Coordonnees haut droit (2 parametres) : (#3,#4)
% Orientation : #5
%   1 : diagonale de pente 1
%  -1 : diagonale de pente -1
%   0 : horizontal
%   2 : vertical
% Nombre de pas horizontaux : #6
% Epaisseur du trait : #7
% Couleur : #8 (ex. green)
% Atténuation couleur : #9 (ex. 30)
\pgfmathsetmacro{\N}{#6-1}
\pgfmathsetmacro{\A}{#1}
\pgfmathsetmacro{\B}{#2}
\pgfmathsetmacro{\C}{#3}
\pgfmathsetmacro{\D}{#4}
\pgfmathsetmacro{\I}{(#3-#1)/#6}
\pgfmathsetmacro{\J}{(#4-#2)/#6}
\ifthenelse{\equal{#5}{1}}{
  \foreach \n in {0,...,\N}
    \foreach \m in {0,...,\N}
      {
        \pgfmathsetmacro{\X}{\A + ((0 + \n) * \I)}
        \pgfmathsetmacro{\Y}{\B + ((0 + \m) * \J)}
        \pgfmathsetmacro{\U}{\A + ((1 + \n) * \I)}
        \pgfmathsetmacro{\V}{\B + ((1 + \m) * \J)}
        \draw[#8!#9,#7] (\X, \Y)--(\U, \V);
      } 
  }{}
\ifthenelse{\equal{#5}{-1}}{
  \foreach \n in {0,...,\N}
    \foreach \m in {0,...,\N}
      {
        \pgfmathsetmacro{\X}{\A + ((1 + \n) * \I)}
        \pgfmathsetmacro{\Y}{\B + ((0 + \m) * \J)}
        \pgfmathsetmacro{\U}{\A + ((0 + \n) * \I)}
        \pgfmathsetmacro{\V}{\B + ((1 + \m) * \J)}
        \draw[#8!#9,#7] (\X, \Y)--(\U, \V);
      } 
  }{}
\ifthenelse{\equal{#5}{0}}{
  \foreach \n in {0,...,\N}
    \foreach \m in {0,...,\N}
      {
        \pgfmathsetmacro{\X}{\A + ((0 + \n) * \I)}
        \pgfmathsetmacro{\Y}{\B + ((0 + \m) * \J)}
        \pgfmathsetmacro{\U}{\A + ((1 + \n) * \I)}
        \pgfmathsetmacro{\V}{\B + ((0 + \m) * \J)}
        \draw[#8!#9,#7] (\X, \Y)--(\U, \V);
      } 
  }{}
\ifthenelse{\equal{#5}{2}}{
  \foreach \n in {0,...,\N}
    \foreach \m in {0,...,\N}
      {
        \pgfmathsetmacro{\X}{\A + ((0 + \n) * \I)}
        \pgfmathsetmacro{\Y}{\B + ((0 + \m) * \J)}
        \pgfmathsetmacro{\U}{\A + ((0 + \n) * \I)}
        \pgfmathsetmacro{\V}{\B + ((1 + \m) * \J)}
        \draw[#8!#9,#7] (\X, \Y)--(\U, \V);
      } 
  }{}
}

%Définition d'un pattern de type hachure
% \usetikzlibrary{patterns}
% \makeatletter
% \tikzset{hatch distance/.store in=\hatchdistance,hatch distance=5pt,hatch thickness/.store in=\hatchthickness,hatch thickness=5pt}

% \pgfdeclarepatternformonly[\hatchdistance,\hatchthickness]{north east hatch}% name
%     {\pgfqpoint{-\hatchthickness}{-\hatchthickness}}% below left
%     {\pgfqpoint{\hatchdistance+\hatchthickness}{\hatchdistance+\hatchthickness}}% above right
%     {\pgfpoint{\hatchdistance}{\hatchdistance}}%
%     {
%         \pgfsetcolor{\tikz@pattern@color}
%         \pgfsetlinewidth{\hatchthickness}
%         \pgfpathmoveto{\pgfqpoint{-\hatchthickness}{-\hatchthickness}}       
%         \pgfpathlineto{\pgfqpoint{\hatchdistance+\hatchthickness}{\hatchdistance+\hatchthickness}}
%         \pgfusepath{stroke}
%     }
% \pgfdeclarepatternformonly[\hatchdistance,\hatchthickness]{north west hatch}% name
%     {\pgfqpoint{-\hatchthickness}{-\hatchthickness}}% below left
%     {\pgfqpoint{\hatchdistance+\hatchthickness}{\hatchdistance+\hatchthickness}}% above right
%     {\pgfpoint{\hatchdistance}{\hatchdistance}}%
%     {
%         \pgfsetcolor{\tikz@pattern@color}
%         \pgfsetlinewidth{\hatchthickness}
%         \pgfpathmoveto{\pgfqpoint{\hatchdistance+\hatchthickness}{-\hatchthickness}}
%         \pgfpathlineto{\pgfqpoint{-\hatchthickness}{\hatchdistance+\hatchthickness}}
%         \pgfusepath{stroke}
%     }
% \makeatother
%~~~~~~~~~~~~~~~~~~~~~~~~~~~~~~~~~~~~~


% Chapter 7
\newcommand\pef[1]{(\ref{#1})}

\newcommand{\subscript}[1]{\textsubscript}
